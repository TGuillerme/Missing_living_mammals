\documentclass[12pt,letterpaper]{article}

%Packages
\usepackage{pdflscape}
\usepackage{fixltx2e}
\usepackage{textcomp}
\usepackage{fullpage}
\usepackage{float}
\usepackage{latexsym}
\usepackage{url}
\usepackage{epsfig}
\usepackage{graphicx}
\usepackage{amssymb}
\usepackage{amsmath}
\usepackage{bm}
\usepackage{array}
\usepackage[version=3]{mhchem}
\usepackage{ifthen}
\usepackage{caption}
\usepackage{hyperref}
\usepackage{amsthm}
\usepackage{amstext}
\usepackage{enumerate}
\usepackage[osf]{mathpazo}
\usepackage{dcolumn}
\usepackage{lineno}
\usepackage{color}
\usepackage[usenames,dvipsnames]{xcolor}
\pagenumbering{arabic}

%Pagination style and stuff
%\linespread{2} 

\raggedright
\setlength{\parindent}{0.5in}
\setcounter{secnumdepth}{0} 
\renewcommand{\section}[1]{%
\bigskip
\begin{center}
\begin{Large}
\normalfont\scshape #1
\medskip
\end{Large}
\end{center}}
\renewcommand{\subsection}[1]{%
\bigskip
\begin{center}
\begin{large}
\normalfont\itshape #1
\end{large}
\end{center}}
\renewcommand{\subsubsection}[1]{%
\vspace{2ex}
\noindent
\textit{#1.}---}
\renewcommand{\tableofcontents}{}

\setlength\parindent{0pt}

\begin{document}

\textbf{RE: Decision on Manuscript ID RSBL-2015-1003}\\
\bigskip
Dear Surayya Johar,\\
\bigskip
We are very grateful to the four referees for their helpful and constructive comments, which we believe have helped us to significantly improve our paper. We have taken all of their comments on board, and respond to their points below. For improving clarity in this document, we dealt with each comment in the order they appeared in your ``Decision on Manuscript'' email and we copied the reviewers comments in blue.

%%%%%%%%%%%%%%%
% Reviewer 1
%%%%%%%%%%%%%%%

\section{Reviewer 1:}
\begin{enumerate}
\item{\textcolor{blue}{In ESM 1, p. 4, the authors explain their methods of collecting morphological data from the literature. Why select only the first 10 results per search term? How do you know that relevant papers wouldn’t come up later in the search? I need evidence to support this cutoff; as it stands, I’m concerned that you’re missing some relevant material.}}

As shown in the ESM1 - Figure 1, the number of OTUs with cladistic data extracted from the Google Scholar searches seems to plateau really quickly (i.e. before the first 100 matches).
Additionally, as Graeme Lloyd mention, ``older studies tend to be smaller (due to computing limitations) and to commonly reoccur anyway due to data set reuse'' (Reviewer 4, comment 1).
Following these two observations, we think that including on the 20th most recent results per order (and higher taxonomic level) was a good balance between the time spending looking for papers, the amount of the OTUs extracted and the repeatability of the study.

\item{\textcolor{blue}{Similarly, I find the 2010 date cutoff troubling, because I’m not sure that the argument that these recent studies contain a fraction of the previously published characters and OTUs allows you to make a comprehensive statement about the availability of character data. What about the fraction that isn’t included? This is a question that needs to be better addressed before you will convince readers that this is really all the data that is out there.}}

%TO DO! Same answer as above?

\end{enumerate}

\subsection{Reviewer 1 specific comments:}
\begin{enumerate}
\item{\textcolor{blue}{P.2, line 43: ``chunks'' is not an appropriate term to use in a scientific paper. Use a more formal term.}}
We replaced ``chunks'' by ``parts'' on p.@, line@.

\item{\textcolor{blue}{P.2, L53 \& L55: ``branch'' is the wrong verb. You want to say something like ``fossils will not be placed accurately within the phylogeny''.}}
We replaced ``fossils cannot branch in the correct clade'' by ``fossils will not be placed accurately within the correct clade'' on p.@, line@.

\item{\textcolor{blue}{P3, L78: ``since'' should be ``because.'' Since should be used for time, not causation.}}
We replaced ``since'' by ``because'' on p.@, line@.

\item{\textcolor{blue}{P3, L85: ``not-applicable'' should not be hyphenated.}}
We removed the hyphen on p.@, line@.

\item{\textcolor{blue}{P3, L120: How much of a difference does the 25\% or less coverage make to the capacity of Total Evidence phylogenetic analyses to resolve relationships? Does your simulation study have a way of quantifying or describing the probability of getting the ``right'' answer with this little data? It would also help if you could show in your data table what the clades were that had better than 75\% coverage. It’s not immediately evident from reading Table 1.}}

% TO DO. To the question, hard to give a value for the probability of getting the ``right'' tree but 25% missing living leads to 0.5 wild card taxa displacement and between 0.8 and 0.4 clade conservation on a linear scale from 1 to 0 with one being the right tree and 0 being a random tree.

\item{\textcolor{blue}{P5, L137: ``taxa'' should be ``OTUs.''}}
We replaced ``taxa'' by ``OTUs'' on p.@, line@.

\item{\textcolor{blue}{P5, L165 \& 166: This sentence also misuses ``branch'' as on P.2. Reword for correct verb usage.}}
We replaced ``unable to branch in the Herpestidae'' by ``unable to be placed in the Herpestidae clade'' on p.@, line@.

\item{\textcolor{blue}{ESM 1, p. 2: ``If some where present'' should be ``if some were present.''}}
We fixed this typo.

\item{\textcolor{blue}{ESM 1, p. 3: ``one of these repository'' should be ``one of these repositories.''}}
We fixed this typo.

\item{\textcolor{blue}{ESM 1, p. 4: ``that were to irrelevant'' should be ``that were too irrelevant.''}}
We fixed this typo.

\item{\textcolor{blue}{ESM 1, p. 4: ``of the recent published matrix'' should be ``of the recently published matrices.''}}
We fixed this typo.

\item{\textcolor{blue}{ESM 1, p. 4: First sentence of the paragraph in the middle of the page is a sentence fragment, and needs to state where the matrices are available.}}
We added the two missing link in the ESM1: ``The list of all the 286 download matrices is available on \url{github.com/TGuillerme/Missing_living_mammals/tree/master/Data/Matrices}.'' and ``All the standardised matrices are available on \url{github.com/TGuillerme/Missing_living_mammals/tree/master/Data/Matrices_binomial/Matrices}.''

\item{\textcolor{blue}{ESM 1, p. 4: ``since the entries in the…'' should be ``because the entries in the…'' Since refers to time, not causation.}}
We fixed this typo.

\item{\textcolor{blue}{ESM 1, p. 5: axis title should be ``Google Scholar matches,'' not ``Google Scholars matches.'' Figure caption should capitalize ``Google Scholar.'' Y-axis and X-axis should be hyphenated.}}
We fixed the typo in the figure and the caption as well as in the code for reproducing this figure.

\item{\textcolor{blue}{ESM 1, p. 6: ``wrong binomial names format'' should be ``incorrectly-formatted binomial names,'' and ``into the correct ones'' should be deleted.
Fixing is assumed to turn incorrect things into correct ones.
Also, the sentence starting the paragraph under ``Selecting the living OTUs'' repeats the exact same thing as this sentence.}}
We modified the sentence to: ``[we] fixed the incorrectly-formatted bionomial names format (e.g. \textit{H. sapiens} became \textit{Homo sapiens})''

\item{\textcolor{blue}{ESM 2 seems to be almost completely redundant with Table 1. Is there some more efficient way of dealing with this information?}}

%TO DO: not sure but maybe we'll just put the 1 character threshold thingy.

\end{enumerate}

%%%%%%%%%%%%%%%
% Reviewer 2 (Graham Slater)
%%%%%%%%%%%%%%%

\section{Reviewer 2:}
\begin{enumerate}
\item{\textcolor{blue}{The authors state on page 3 lines 82-90 that they avoided matrices with ``few characters'' due to lower probability of overlapping characters with other matrices and so only selected matrices containing >100 characters. [...] Is 100 really the minimum number you want to go with at all phylogenetic scales?}}

%TO DO: let's just remove this threshold thingy since I seem to be the only one to be convinced by it ;)...


\item{\textcolor{blue}{I finally a little skeptical of the sampling done. Looking at Figure 1B, again the clade I’m most familiar with, representation for carnivorans appears very poor. Notwithstanding the issues discussed above regarding matrices with < 100 characters, there are definitely matrices out there that fulfill the authors’ requirements that are not represented. For example, a 349 character matrix with near complete sampling at the species is available for Viverridae, Prionodontidae, and Eupleridae in Gaubert et al (2005 Syst Biol.). Going through each clade in the supplement seems beyond the scope of reviewing duties, but I am concerned that if this one matrix can be missed others may have been missed too. I think a reexamination of the literature may be warranted using taxon specific search terms.}}

%TO DO: OK, I think there is a problem here. Either we have to rerun the matrix downloading procedure and be sure we include the ones specified by the reviewers (the correct way!) but that WILL take more than a week. Or else we just add the matrices that are suggested y the reviewers but I have the feeling that that is a bit of cheating (i.e. we're hand picking some data here) - basically that'll just add more carnivora.
%I think one argument could be to emphasize on one of the points we add in a former version of the manuscript (can't find it): "by available data we mean data in a standard format that can be directly downloaded from the internet (i.e. many of the papers we looked at had no link to their data or the link was broken, meaning that the data is effectively only available upon specific request to the authors [baaaad!!!])."

\item{\textcolor{blue}{One interesting aside that the authors didn’t consider, despite the data being there, is how phylogenetic ``effort'' relates to diversity of the order under study.}}

We are really grateful for Graham Slater's %or do we need to keep it formal and say just "reviewer 2's"
comment and added this interesting analysis in the ESM @@@% get number.
% that mammalian morphological phylogeneticists have focused their energies on resolving the higher level relationships among mammalian lineages.
\end{enumerate}

%%%%%%%%%%%%%%%
% Reviewer 3 (Peter Wagner)
%%%%%%%%%%%%%%%

\section{Reviewer 3:}
\begin{enumerate}
\item{\textcolor{blue}{My biggest criticism is a semantic one: Morphological data $\neq$ cladistic data.}}
\item{\textcolor{blue}{The real issue here is the availability of phylogenetic character coding for anatomical data.
[...] Thus, the title really should be ``Assessment of available anatomical characters for phylogenetic analysis among living mammals'' or something like that.}}
\item{\textcolor{blue}{what we really need is to focus on the tooth and skull characters that typify the fossil record.
[...] Still, several papers estimate that the sampling of mammal species (based on cranial characters include teeth) actually is pretty darned good [...].
Thus, we might be able to richly populate FBD and tip-dating analyses with a LOT of extinct species.}}
\item{\textcolor{blue}{Finally, I would add: who the heck submits a paper for a special volume 6 weeks early?!?!? You make the rest of us look bad!}}
This paper was actually ready to submit from the 28th of July (\url{http://biorxiv.org/content/early/2015/07/28/022970}) which partially explain the swift submission.
We do sincerely apologize for making the rest of the authors submitting to the issue look bad.

\end{enumerate}

\subsection{Reviewer 3 specific comments:}
\begin{enumerate}
\item{\textcolor{blue}{Lines 25-26: ``We suggest that increased morphological data collection efforts for living taxa are needed to produce accurate Total Evidence phylogenies.''
In a way, this is more than ``total evidence.=''.
The initial total evidence paradigm was simply to use every character you could (molecular, morphological, even behavioral).
We now are talking about tip-dating and fossilized birth-death (FBD) approaches (e.g., Heath et al. 2014 PNAS 111:E2957) that want to use all of the available data.
After all, one can do a total evidence analysis with just node-dating techniques rather than with tip-dating, and one can do it without worrying about branch length priors.
However: if we want to use all of the data to calibrate branch lengths for all taxa (fossil and extant), then we need ``total evidence'' to do tip-dating and FBD properly).
As tip-dating is the main theme of this issue, that should be linked to the total evidence explicitly here and elsewhere in the manuscript.}}

\item{\textcolor{blue}{Lines 70-71: ``We downloaded all cladistic matrices containing any living and/or fossil mammal  taxa from three major public databases:''
As a side note and for future reference for the authors, a large number of phylogenetic data sets for mammals (and other metazoans) also are available at the Paleobiology Database.
(These were formerly at my website at the Field Museum, but that site obviously disappeared when I moved to the Smithsonian.)
Because of the nature of the projects for which I was using these datasets, most of them are species-level, but some are genus-level, too.}}

\item{\textcolor{blue}{Lines 164-166: ``For example, a Carnivora fossil will be unable to branch in the Herpestidae, and will have more chance to randomly branch within Canidae (Figure 1B).'' 
This seems to be an awkward way to phrase this.
Moreover, this seems like an extension of the long-branch problem.
Basically, the issue really is that we do not have coded skeletal characters for extant herpestids.
Thus, it is not possible for coded skeletal characters from fossil herpestids to connect those fossil herpestids to the extant herpestids.
This is the opposite of Felsenstein’s long branch: insofar as skeletal characters are concerned, extant herpestids have zero-length branches.
That in turn means that the fossil herpestids are going to be ``attracted'' to non-zero length branches elsewhere, with long branches in related groups such as canids being particularly good candidates because there is a higher chance of homoplasies between canids and herpesitds on the longer branches.
So, it will not be random: they probably will latch themselves onto the longest reasonably branch among the closest relatives for which anatomical data are present.}}

\end{enumerate}

%%%%%%%%%%%%%%%
% Reviewer 4 (Graeme Lloyd)
%%%%%%%%%%%%%%%

\section{Reviewer 4:}
\begin{enumerate}
\item{\textcolor{blue}{Reviewer 1’s most substantial complaints (the year and page number cut offs for google scholar searches) seem at least partially valid.
However, in my experience older studies tend to be smaller (due to computing limitations) and to commonly reoccur anyway due to data set reuse.
The authors also already state that subsequent search results tended to yield no substantial additional data, and by the reviewer’s own admission the assembled data seem to look correct.
Thus a recollecting of the data doesn’t seem necessary (but see further comments below).}}

\item{\textcolor{blue}{Reviewer 2 (Graham Slater)’s complaint are - in my view - the most substantial.
I agree that the >100 characters cut off is not as strongly justified as it might be.
However, as the authors have run their protocol without this threshold this is easily fixed by at least briefly mentioning how this affects their results in the main text (not relegating it to the ESM where no actual discussion is provided).
From my reading of the two tables only 11 of the orders seem to have below 25\% coverage at the species-level (not 22 as the authors state in the main text!) when the threshold is used, and this improves slightly to 10 for when the threshold is removed.
As this difference isn’t dramatic it seems worth mentioning.
Reviewer 2 also notes that some important studies may be missed.
This seems valid (I am aware of the biases in Ross Mounce’s data collection protocol, and obviously my own) and I am sure there are many more data sets that could be sampled.
In particular I am aware of a number of studies that can only be found in books and hence are not easily accessed even if they occur in a google scholar search.
I do not expect the authors to more thoroughly search the literature for all possible cladistic analyses of mammals, however, as this is a major task.
(I should know, it has been a long term project of mine for years.)
Nevertheless, it does deserve mention in the discussion as a potential bias in the results.
I don’t think this undermines the study, but rather should be viewed as both a reason for optimism and an acknowledgement of a separate problem (i.e., access to cladistic data) which multiple publications already cover (e.g., see Ross Mounce’s work).}}

\item{\textcolor{blue}{Reviewer 3 (Pete Wagner)’s major problem - with which I agree - is semantic, and thus easily fixable with a few explicit caveats (e.g., ``by cladistic we mean...'').}}

\item{\textcolor{blue}{One point I would add of my own is that the orders used are obviously very variable in size (species number), so what about total coverage?
This could surely be synthesised from the tables and would be a nice simple way to compare the with and without threshold results.
Similarly, I think Reviewer 3’s [We assume Graeme Lloyd is refering to Graham Slater's reviewe (reviewer 2)] %or same as comment before: no names!
 graphs are a nice potential addition to the ESM.}}
\end{enumerate}

\subsection{Reviewer 4 specific comments:}
\begin{enumerate}
\item{\textcolor{blue}{Lines 72-74 - Web addresses should be consistent in dropping of ``http://www.''.}}
We removed the ``http://www.'' part of the URLs on p.@, lines@ and in the ESM1.

\item{\textcolor{blue}{Line 104 - ``NTI 15'' should probably be ``NTI [15]''.}}
%TO DO IN THE WORD MANUSCRIPT! We fixed this typo.

\item{\textcolor{blue}{Line 104 - It’s not completely clear to me how ``distance'' is measured. Is this branch-lengths on a time-scaled tree? Or just a node count?}}
We changed ``$\overline{MNND}_{obs}$ is the observed mean distance between each of $n$ taxa'' to ``$\overline{MNND}_{obs}$ is the observed mean branch length between each of $n$ taxa'' and ``$\overline{MPD}_{obs}$ is the observed mean phylogenetic distance of the tree containing only the $n$ taxa'' to ``$\overline{MPD}_{obs}$ is the observed mean phylogenetic branch length of the tree containing only the $n$ taxa'' on p.@, lines@.

\item{\textcolor{blue}{Line 117 - It’s not clear to me how supraspecific values were calculated if the reference phylogeny (ref 10) is species-level. Are you using the node subtending the higher taxon or something else?}}
%TO CHECK!

\item{\textcolor{blue}{Line 149 - ``most palaeontological studies use the genus as their smallest OTU'' I’m not sure if this is strictly true. It’s more that shorter OTU names are commonly used. After all at least one actual specimen has to be coded and in my experience if you read the paper usually only a single species is used. Palaeontologists are just bad at using proper binomials in their matrices (which is an issue, but a different one). In addition ref 20 is actually all species-level OTUs! (I didn’t check ref 21.)}}
%To CHECK, do we actually care?

\item{\textcolor{blue}{Line 154 - ``phylogenetically overdispersed'' Purely semantic, but I guess what you mean is that even sampling would be ideal, and this is *measured* using phylogenetic over-dispersion.}}
We changed ``the ideal scenario is for them to be phylogenetically overdispersed'' to ``the ideal scenario is for them to be evenly distributed (as measured by phylogenetic overdispersion)'' on p.@, line@.

\item{\textcolor{blue}{Line 245 - The caption could be more informative regarding the random and clustered distributions that these figures represent.}}
We changed the caption to: ``Phylogenetic distribution of species with available cladistic data across two orders (A: Primates; B: Carnivora). Blue branches indicate available cladistic data for the species. Cladistic data is randomly distributed in Primates (A) and clustered in Carnivora (B).''.

\item{\textcolor{blue}{ESM 1 - Incomplete sentence: ``The list of all the 286 download matrices is available on''}}
See reviewer 1 specific comment 12.

\item{\textcolor{blue}{ESM 1 - Incomplete sentence: ``All the standardised matrices are available on''}}
See reviewer 1 specific comment 12.

\item{\textcolor{blue}{ESM 1 - ``designated as ``fossil'' all OTUs that were present in the Paleobiology database'' The database contains extant taxa too, although these should be marked as such. I guess this doesn’t matter given the flow chart in Figure 2, but is something the authors should be aware of. (Obviously more generally living species can have fossil records, or at least plenty of fossils are assigned to extant species.)}}
%TO DO: change ``fossil'' to ``non living OTU''? As in OTU that is not present in the Wilson Reeder list or the Bininda Emonds tree?

\item{\textcolor{blue}{ESM 1 Figure 2 - How many OTUs were ignored? And based on the (assumed) cursory inspection of the authors what are they? Higher taxa? Specimens? Or just species you can’t reconcile with your three data sets?}}
%TO DO: add the number of ignored OTUs. I think they're mainly specimen IDs without species names.

\item{\textcolor{blue}{ESM 2 - ``The following section contains supplementary results to the main body: the available data structure using the NTI and the PD metric'' Abbreviations should be introduced when first used, not later as they are here.}}
We changed ``using the NTI and the PD metric'' to ``using the Net Relatedness Index (NRI) and the Nearest Taxon Index (NTI) metrics''.
\end{enumerate}



We hope we have responded to all these comments appropriately. Please let us know if you require any further information,\\
\bigskip

% Editorial office comments to authors:
% Please ensure that you include; 
% *An author contributions section which fulfils all *four* of our criteria given here https://royalsociety.org/journals/ethics-policies/openness/, including a sentence stating all authors agree to be held accountable for the content therein and approve the final version of the manuscript 
% *A funding statement which is the same as that given in ScholarOne
% *Please also confirm in your cover letter whether all the figures are your own or whether permission has been obtained for their use
% *Please upload your original figure files as eps, tiff or jpeg files, rather than PDFs
% *If you have any images that can be used to promote your article on social media (should this be accepted) please upload them as a supplementary file
% *Please confirm whether your manuscript should be available via open access, as we note that this may be a requirement of your funder.


Thomas Guillerme (guillert@tcd.ie)\\ % Changing email + adding imperial address?
Natalie Cooper

\end{document}