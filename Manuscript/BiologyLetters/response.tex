\documentclass[12pt,letterpaper]{article}

%Packages
\usepackage{pdflscape}
\usepackage{fixltx2e}
\usepackage{textcomp}
\usepackage{fullpage}
\usepackage{float}
\usepackage{latexsym}
\usepackage{url}
\usepackage{epsfig}
\usepackage{graphicx}
\usepackage{amssymb}
\usepackage{amsmath}
\usepackage{bm}
\usepackage{array}
\usepackage[version=3]{mhchem}
\usepackage{ifthen}
\usepackage{caption}
\usepackage{hyperref}
\usepackage{amsthm}
\usepackage{amstext}
\usepackage{enumerate}
\usepackage[osf]{mathpazo}
\usepackage{dcolumn}
\usepackage{lineno}
\usepackage{color}
\usepackage[usenames,dvipsnames]{xcolor}
\pagenumbering{arabic}

%Pagination style and stuff
%\linespread{2} 

\raggedright
\setlength{\parindent}{0.5in}
\setcounter{secnumdepth}{0} 
\renewcommand{\section}[1]{%
\bigskip
\begin{center}
\begin{Large}
\normalfont\scshape #1
\medskip
\end{Large}
\end{center}}
\renewcommand{\subsection}[1]{%
\bigskip
\begin{center}
\begin{large}
\normalfont\itshape #1
\end{large}
\end{center}}
\renewcommand{\subsubsection}[1]{%
\vspace{2ex}
\noindent
\textit{#1.}---}
\renewcommand{\tableofcontents}{}

\setlength\parindent{0pt}

\begin{document}

\textbf{RE: Decision on Manuscript ID RSBL-2015-1003}\\
\bigskip
Dear Surayya Johar,\\
\bigskip
We are very grateful to the four referees for their helpful and constructive comments.
We have taken all of their comments on board, and respond to their points below. To improve the clarity of this document, we dealt with each comment in the order they appeared in your ``Decision on Manuscript'' email and we have included the reviewers comments in blue.

%TG: plus response to editors points:
% *Please also confirm in your cover letter whether all the figures are your own or whether permission has been obtained for their use
% *Please confirm whether your manuscript should be available via open access, as we note that this may be a requirement of your funder.
We would like to state that all the figures are ore our own and the the manuscript should be available via open access.
% *If you have any images that can be used to promote your article on social media (should this be accepted) please upload them as a supplementary file
Additionally, we attached a supplementary file named \texttt{article\_promotion.eps} for promoting our article (should it be accepted) on social media. 

% *Please upload your original figure files as eps, tiff or jpeg files, rather than PDFs
%TG: TO DO (easy)


%%%%%%%%%%%%%%%
% Reviewer 1
%%%%%%%%%%%%%%%

\section{Reviewer 1:}
\begin{enumerate}
\item{\textcolor{blue}{In ESM 1, p. 4, the authors explain their methods of collecting morphological data from the literature. Why select only the first 10 results per search term? How do you know that relevant papers wouldn't come up later in the search? I need evidence to support this cutoff; as it stands, I'm concerned that you're missing some relevant material.}}
% NC: It's not just the first 10 results per search term is it? I thought you went through the first 10 pages of results? Clarify this. TG: No I collected the 20 first results. It's mentioned down there.

As shown in ESM1 Figure 1, the number of OTUs with morphological cladistic data extracted from the Google Scholar searches seems to plateau really quickly (i.e. before the first 100 matches).
Additionally, as Graeme Lloyd mentions, ``older studies tend to be smaller (due to computing limitations) and to commonly reoccur anyway due to data set reuse'' (Reviewer 4, comment 1).
Following these two observations, we think that including the 20 most recent results for each order (and higher taxonomic level) was a good balance between the time spent looking for papers, the number of OTUs extracted and the repeatability of the study.
We are aware however that we might have missed some specific studies (see Reviewer 2 comment 2) but are confident that our protocol gives a good overall picture of the data availability and that the three repositories used before the Google Scholar search (MorphoBank, Graeme Lloyd's and Ross Mounce's) contained most of the available data.
In fact, as mentioned in ESM 1, part 1 ``among the 660 papers [collected from the Google Scholar search], only 50 contained a total of 425 extra living OTUs''. % NC: How many OTUs were in the other 610 papers?

\item{\textcolor{blue}{Similarly, I find the 2010 date cutoff troubling, because I'm not sure that the argument that these recent studies contain a fraction of the previously published characters and OTUs allows you to make a comprehensive statement about the availability of character data. What about the fraction that isn't included? This is a question that needs to be better addressed before you will convince readers that this is really all the data that is out there.}}

We chose this cutoff date to filter our analysis towards accessible data only (see Reviewer 2 comment 2).
Additionally, we observed that in newer studies (i.e. post 2010), living taxa with anatomical data were rarely added and usually reused from former studies (with the occasional addition of extra characters and extra living taxa).
We modified the sentence in the ESM1 part 1:

``For example in primates the character p7 coded first by [1] is reused with the same living species in [2], [3], [4], [5], [5], [6], [7], [8], [9], [10], [11], [12] and [13].''

\noindent to: ``For example, the six living primates used in [1] (\textit{Aotus trivirgatus, Galago demidoff, Lemur catta, Microcebus murinus, Nycticebus coucang and Saimiri sciureus}) and their associated characters are reused along with more living species and characters in [2], [3], [4], [5], [5], [6], [7], [8], [9], [10], [11], [12] and [13].''.

\end{enumerate}
\subsection{Reviewer 1 specific comments:}
\begin{enumerate}
\item{\textcolor{blue}{P.2, line 43: ``chunks'' is not an appropriate term to use in a scientific paper. Use a more formal term.}}
We replaced ``chunks'' with ``sections''.

\item{\textcolor{blue}{P.2, L53 \& L55: ``branch'' is the wrong verb. You want to say something like ``fossils will not be placed accurately within the phylogeny''.}}
We replaced ``fossils cannot branch in the correct clade'' with ``fossils will not be placed accurately within the correct clade''.

\item{\textcolor{blue}{P3, L78: ``since'' should be ``because.'' Since should be used for time, not causation.}}
We replaced ``since'' with ``because'' throughout the manuscript as appropriate.

\item{\textcolor{blue}{P3, L85: ``not-applicable'' should not be hyphenated.}}
We removed the hyphen.

\item{\textcolor{blue}{P3, L120: How much of a difference does the 25\% or less coverage make to the capacity of Total Evidence phylogenetic analyses to resolve relationships? Does your simulation study have a way of quantifying or describing the probability of getting the ``right'' answer with this little data? It would also help if you could show in your data table what the clades were that had better than 75\% coverage. It's not immediately evident from reading Table 1.}}

The relationship between the living species coverage and the Total Evidence phylogeny topology is hard to quantify precisely and depends on both the metric used for measuring topological differences and the method used to infer the tree.
Our simulations showed that topology tends to get closer to random as morphological data is removed but that in the case of missing data for living species, the other parameters we tested (missing data in the fossil record and number of morphological characters) had no significant effect (i.e. when a certain number of living taxa are missing, the other parameters do not have any extra influence).
In our simulations we tested five different levels of missing living taxa (0\%, 10\%, 25\%, 50\% and 75\% missing living taxa) and we observed that the three first levels had no significant negative effect on topology in relation with the amount of missing data in the fossil record and the number of morphological characters but that the fifth level (75\% of missing living taxa, i.e. 25\% coverage) had a strong negative effect.

Also, we highlighted the different levels of coverage in Table 1 as follow: at any taxonomic level, orders with less than 25\% coverage are highlighted in blue (or dark grey in the black and white version) and orders with more than 75\% coverage are highlighted in orange (or light grey in the black and white version).

\item{\textcolor{blue}{P5, L137: ``taxa'' should be ``OTUs.''}}
We replaced ``taxa'' with ``OTUs''

\item{\textcolor{blue}{P5, L165 \& 166: This sentence also misuses ``branch'' as on P.2. Reword for correct verb usage.}}
We replaced ``unable to branch in the Herpestidae'' with ``unable to be placed in the Herpestidae clade''.

\item{\textcolor{blue}{ESM 1, p. 2: ``If some where present'' should be ``if some were present.''}}
We fixed this typo.

\item{\textcolor{blue}{ESM 1, p. 3: ``one of these repository'' should be ``one of these repositories.''}}
We fixed this typo.

\item{\textcolor{blue}{ESM 1, p. 4: ``that were to irrelevant'' should be ``that were too irrelevant.''}}
We fixed this typo.

\item{\textcolor{blue}{ESM 1, p. 4: ``of the recent published matrix'' should be ``of the recently published matrices.''}}
We fixed this typo.

\item{\textcolor{blue}{ESM 1, p. 4: First sentence of the paragraph in the middle of the page is a sentence fragment, and needs to state where the matrices are available.}}
We added the two missing links in the ESM1: ``The list of all the 286 download matrices is available on \url{github.com/TGuillerme/Missing_living_mammals/tree/master/Data/Matrices}.'' and ``All the standardised matrices are available on \url{github.com/TGuillerme/Missing_living_mammals/tree/master/Data/Matrices_binomial/Matrices}.''

\item{\textcolor{blue}{ESM 1, p. 4: ``since the entries in the…'' should be ``because the entries in the…'' Since refers to time, not causation.}}
We fixed this typo.

\item{\textcolor{blue}{ESM 1, p. 5: axis title should be ``Google Scholar matches,'' not ``Google Scholars matches.'' Figure caption should capitalize ``Google Scholar.'' Y-axis and X-axis should be hyphenated.}}
We fixed the typo in the figure and the caption as well as in the code for reproducing this figure.

\item{\textcolor{blue}{ESM 1, p. 6: ``wrong binomial names format'' should be ``incorrectly-formatted binomial names,'' and ``into the correct ones'' should be deleted.
Fixing is assumed to turn incorrect things into correct ones.
Also, the sentence starting the paragraph under ``Selecting the living OTUs'' repeats the exact same thing as this sentence.}}
We modified the sentence to: ``[we] fixed the incorrectly-formatted binomial names (e.g. \textit{H. sapiens} became \textit{Homo sapiens})''.

\item{\textcolor{blue}{ESM 2 seems to be almost completely redundant with Table 1. Is there some more efficient way of dealing with this information?}}

We replaced the Table 1 in the main text with the Table 1 in the ESM 2 (available data without any number of characters threshold - see reviewer 2 comment 1).
We therefore removed ESM 2 and renamed ESM 3 to ESM 2

\end{enumerate}

%%%%%%%%%%%%%%%
% Reviewer 2 (Graham Slater)
%%%%%%%%%%%%%%%

\section{Reviewer 2 (Graham Slater):}
\begin{enumerate}
\item{\textcolor{blue}{The authors state on page 3 lines 82-90 that they avoided matrices with ``few characters'' due to lower probability of overlapping characters with other matrices and so only selected matrices containing >100 characters. First, a lack of overlapping characters doesn’t seem justifiable, at least to my mind, as a reason to exclude a matrix if the ultimate goal is to infer the placement of extinct taxa relative to extant taxa. Antler ramifications, to use the authors' example, are likely to be a very useful character when attempting to resolve relationships among living and extinct artiodactyls, even if they cannot be coded for monotremes. To flip the argument, a matrix of 200 + characters for mysticetes is similarly likely to contain many non-applicable characters from the perspective of talpids. I do understand that the authors are thinking downstream in terms of super matrices for higher level analyses, but smaller matrices are going to be just as relevant for reaching this goal. You don't have much space in the ms I understand, but surely this needs to be addressed?!}}

We completely see the reviewer's point and have removed this threshold and replaced it with the analysis without any minimal number of morphological characters.
We also removed the paragraph about the character threshold analysis and updated the results of figure 1 and table 1 to contain all living OTUs; we removed the redundant ESM 2 and we updated the figures. 

\item{\textcolor{blue}{Along these lines, a 100 character limit may seem rational but this is actually, from a practical perspective, very large. molecular matrices of this size would, of course, be laughed at but many of the most robust phylogenetic studies I'm familiar with come in well under this size. I suspect, but cannot be sure, that this partially explains the poor species level coverage the authors find; higher level studies can readily accumulate large numbers of characters because vastly different morphologies are being compared, often using families or genera as exemplars, leading to an abundance of easily defined characters. Species level studies, on the other, are often focused within families or genera to make them more manageable, but here characters are quickly exhausted. To give some random examples that I'm familiar with, Salles (1992) species-level study of Felidae (all species) was based on 44 characters, Wang and Carranza- Casta\~{n}eda's 2008 study of Mephitidae (\textasciitilde50\% of species but all genera) was based on 38 characters, and Bryant et al's 1993 study of Mustelid genera was of 46 characters. These are all important studies, but do not feature here because of the 100 character limit. At the very least I think this results in a downplaying of previous phylogenetic efforts by morphologists in the results. At worst it misrepresents the state of affairs dramatically. Is 100 really the minimum number you want to go with at all phylogenetic scales?}}

Again we completely see the reviewer's point and have modified our manuscript as follows.
By removing the threshold (see above) we now included the taxa coded in Wang and Carranza-Casta\~{n}eda 2008 (matrix reference GS2014-WCC) and other matrices excluded in the former analysis increasing the number of Carnivora from 42 species (14\% of total number of species) to 76 (27\%).
However, the two other matrices mentioned by reviewer 2 (Salles 1992 and Bryant et al. 1993) were not included in the new analysis.
This is due to two main reasons: firstly, these two matrices were not available on any of the repositories (either Graeme Lloyd's, Ross Mounce's or MorphoBank's), and secondly, they didn't appear in our additional Google Scholar searches because they were published before 2010.
Following this reviewer's comment as well as reviewer 4's we added a paragraph in the discussion to clearly state that we are aware of data sampling biases:

``It is worth noting, however, that our analysis did not include all the matrices containing anatomical characters ever published.
In fact, our data collection procedure focused on including studies that provided matrices easily accessible, i.e. we did specifically not include any matrices that were only available in paper format (e.g. printed in books), non-reusable format (e.g. an image of the matrix) %TG: yes... There were quiet a few of them!
or/and matrices available only upon request (e.g. by emailing the authors).
Such matrices are likely to not be included in Total Evidence analysis due to the difficulty of obtaining them. % NC: I think this last line is a bit dangerous. Anyone making a proper phylogeny would include these, especially for a smaller group. It's what people like Graham spend years doing. Perhaps rephrase: "The difficulty of obtaining such matrices makes them less likely to be included in Total Evidence analyses, especially where the number of taxa to include in the phylogeny is high."
'' p.@ lines@@.


\item{\textcolor{blue}{I finally am a little skeptical of the sampling done. Looking at Figure 1B, again the clade I'm most familiar with, representation for carnivorans appears very poor. Notwithstanding the issues discussed above regarding matrices with < 100 characters, there are definitely matrices out there that fulfill the authors' requirements that are not represented. For example, a 349 character matrix with near complete sampling at the species is available for Viverridae, Prionodontidae, and Eupleridae in Gaubert et al (2005 Syst Biol.). Going through each clade in the supplement seems beyond the scope of reviewing duties, but I am concerned that if this one matrix can be missed others may have been missed too. I think a reexamination of the literature may be warranted using taxon specific search terms.}}

We dealt with this comment by removing the matrix size threshold and adding a paragraph to the discussion explaining that some matrices might not be included in this study (see reviewer's 2 comments 1 and 2).
Concerning the Gaubert et al. 2005's study specifically, this study unfortunately did not contain any easily available matrix.
In fact, the matrix mentioned as Appendix 2 in the paper is not present in the article and neither of the TreeBase accession numbers (``S1255'', ``M2193'') nor the provided URL (``www.systematicbiology.org'') links to the matrix.
% TG: I don't know if that's a good justification but basically, it's impossible to get this specific matrix.
% NC: Hmmm it's tricky as I said above. If you were really building a tree you would find this kind of data somehow. Also wouldn't this not appear because you used the 2010 cutoff? I think tbh Graham has some good points - you did this the quick and dirty way rather than the thorough way someone would do it if they were constructing a tree.

\item{\textcolor{blue}{One interesting aside that the authors didn't consider, despite the data being there, is how phylogenetic ``effort'' relates to diversity of the order under study.}}

This is a great suggestion, thanks!
We added this analysis to the ESM 2 (Supplementary results - 1) showing, as suggested, that mammalian morphological phylogeneticists have focused their energies on resolving the higher level relationships among mammalian lineages.
We also added a sentence mentioning these results in the manuscript:

``It is also encouraging to note that the sampling effort is consistent with diversity at each taxonomic level and there is therefore no direct bias in sampling at any taxonomic level (ESM 2, Figure 1).
'' p.@ lines@@

% NC: OK this is not what Graham meant. I think what he is referring to is that when you have more people working on a group (effort) you are likely to get more taxa with coded data. It's like in taxonomy where groups with more taxonomists (eg primates) have more species than groups with fewer (eg rodents). I think what he'd want is a regression of number of papers (effort) against percentage of coded species/OTUs with each order being a datapoint.

\end{enumerate}

%%%%%%%%%%%%%%%
% Reviewer 3 (Peter Wagner)
%%%%%%%%%%%%%%%

\section{Reviewer 3 (Peter Wagner):}
\begin{enumerate}
\item{\textcolor{blue}{My biggest criticism is a semantic one: Morphological data $\neq$ cladistic data.}}
We agree with this reviewers semantinc point and changed ``cladistic'' and ``morphological'' data to ``anatomical characters'' throughout the manuscript.
% NC: all he is saying is cladistic is not morphological and shouldn't be used as proxies for each other. And that really when we say morphological we mean phylogenetically coded anatomical/morphological characters. I've changed a few of these back in the manuscript to try and make it a bit easier to understand, as anatomical suggests something else I think.

\item{\textcolor{blue}{The real issue here is the availability of phylogenetic character coding for anatomical data.
[...] Thus, the title really should be ``Assessment of available anatomical characters for phylogenetic analysis among living mammals'' or something like that.}}
We changed the title as the reviewer suggested.

\item{\textcolor{blue}{what we really need is to focus on the tooth and skull characters that typify the fossil record.
[...] Still, several papers estimate that the sampling of mammal species (based on cranial characters include teeth) actually is pretty darned good [...].
Thus, we might be able to richly populate FBD and tip-dating analyses with a LOT of extinct species.}}

We totally agree with this comment.
In fact, among the 286 matrices used in this analysis, we found 5228 unique OTUs including 3627 fossil ones which is pretty impressive knowing the amount of work required to collect data for a single specimen!
However, in this study, we focused only on the number of *living* taxa (1601 unique OTUs containing 847 species) as a way to solve the problems associated with the lack of living taxa on tree topology described in Guillerme \& Cooper 2016 (essentially that if you don't have overlapping coded anatomical data for your living species then there is a low likelihood of correctly placing the fossil taxa no matter how good the sampling is for those species). 

\item{\textcolor{blue}{Finally, I would add: who the heck submits a paper for a special volume 6 weeks early?!?!? You make the rest of us look bad!}}
This paper formed part of my PhD thesis that I submitted in September which partially explains the swift submission!
Sorry for making the rest of the authors submitting to the issue look bad.

\end{enumerate}

\subsection{Reviewer 3 specific comments:}
\begin{enumerate}
\item{\textcolor{blue}{Lines 25-26: ``We suggest that increased morphological data collection efforts for living taxa are needed to produce accurate Total Evidence phylogenies.''
In a way, this is more than ``total evidence''.
The initial total evidence paradigm was simply to use every character you could (molecular, morphological, even behavioral).
We now are talking about tip-dating and fossilized birth-death (FBD) approaches (e.g., Heath et al. 2014 PNAS 111:E2957) that want to use all of the available data.
After all, one can do a total evidence analysis with just node-dating techniques rather than with tip-dating, and one can do it without worrying about branch length priors.
However: if we want to use all of the data to calibrate branch lengths for all taxa (fossil and extant), then we need ``total evidence'' to do tip-dating and FBD properly).
As tip-dating is the main theme of this issue, that should be linked to the total evidence explicitly here and elsewhere in the manuscript.}}

We are aware of the difference between the ``Total Evidence method'' (i.e. a way to combine data in a matrix like supermatrix) and ``tip-dating method'' (i.e. a phylogenetic dating method, like node dating).
Both methods can (and are often) be used together as in Ronquist et al 2012 (Systematics Biology) but we are aware that they can also be used really well separately.
However, this study focuses \textit{indeed} on the ``Total Evidence method'' (i.e. a way to combine molecular and discrete anatomical data together) and not on the ``tip-dating'' one because it attempts to quantify the problem of missing data in living taxa underlined in Guillerme \& Cooper 2016 (again, a paper focusing on the ``Total Evidence method'', not including ``tip-dating'' nor ``node dating'' analysis).
Nonetheless, we do understand the reviewer's point and added the mention of ``Total Evidence tip-dated phylogenies'' which is undoubtedly a great avenue of application of both methods.

We modified the sentence in the abstract as follows: ``We suggest that increased morphological data collection efforts for living taxa are needed to produce accurate Total Evidence tip-dated phylogenies.'' p@ lines@@

And clarified the third sentence of the introduction as follows:
``One increasingly popular method, the Total Evidence method [3], combines molecular data from living taxa and morphological data from both living and fossil taxa in a supermatrix that can then be used with the tip-dating method (e.g. [4, 3, 5, 1, 6]), producing a chronogram with living and fossil taxa at the tips.'' p@ lines@@

\item{\textcolor{blue}{Lines 70-71: ``We downloaded all cladistic matrices containing any living and/or fossil mammal  taxa from three major public databases:''
As a side note and for future reference for the authors, a large number of phylogenetic data sets for mammals (and other metazoans) also are available at the Paleobiology Database.
(These were formerly at my website at the Field Museum, but that site obviously disappeared when I moved to the Smithsonian.)
Because of the nature of the projects for which I was using these datasets, most of them are species-level, but some are genus-level, too.}}

We thank the reviewer for this mention and we will be sure to use this database for future studies.
% NC: If you end up needing to redo the data collection it might be worth checking the PBDB...

\item{\textcolor{blue}{Lines 164-166: ``For example, a Carnivora fossil will be unable to branch in the Herpestidae, and will have more chance to randomly branch within Canidae (Figure 1B).'' 
This seems to be an awkward way to phrase this.
Moreover, this seems like an extension of the long-branch problem.
Basically, the issue really is that we do not have coded skeletal characters for extant herpestids.
Thus, it is not possible for coded skeletal characters from fossil herpestids to connect those fossil herpestids to the extant herpestids.
This is the opposite of Felsenstein's long branch: insofar as skeletal characters are concerned, extant herpestids have zero-length branches.
That in turn means that the fossil herpestids are going to be ``attracted'' to non-zero length branches elsewhere, with long branches in related groups such as canids being particularly good candidates because there is a higher chance of homoplasies between canids and herpesitds on the longer branches.
So, it will not be random: they probably will latch themselves onto the longest reasonably branch among the closest relatives for which anatomical data are present.}}

We agree with this comment and changed this to:
``For example, a Carnivora fossil will be unable to be placed in the Herpestidae clade because they have no anatomical characters available (i.e. extant Herpestidae have a zero-length branch for anatomical characters).
Such fossils, however, will have a higher probability of being placed on a branch that contains many anatomical characters (i.e. long branches for anatomical characters, e.g. Canidae or Ursidae (Figure 1B).''

% NC: Yep he's right, modify the sentence to account for this. Also cite a long branch attraction stuff. Check if Pete has written anything - it'd be nice to cite him as he's been so helpful. % TG: I looked for relevant paper on this "short-branch repulsion" but didn't found any that's mildly relevant from Peter...

\end{enumerate}

%%%%%%%%%%%%%%%
% Reviewer 4 (Graeme Lloyd)
%%%%%%%%%%%%%%%

\section{Reviewer 4:}
\begin{enumerate}
\item{\textcolor{blue}{Reviewer 1's most substantial complaints (the year and page number cut offs for google scholar searches) seem at least partially valid.
However, in my experience older studies tend to be smaller (due to computing limitations) and to commonly reoccur anyway due to data set reuse.
The authors also already state that subsequent search results tended to yield no substantial additional data, and by the reviewer's own admission the assembled data seem to look correct.
Thus a recollecting of the data doesn't seem necessary (but see further comments below).}}

We developed the description of our data collection protocol in the ESM 1 (see Reviewer 1 comment 2) and underlined the caveats in the discussion (see Reviewer 2 comment 2).

\item{\textcolor{blue}{Reviewer 2 (Graham Slater)'s complaint are - in my view - the most substantial.
I agree that the >100 characters cut off is not as strongly justified as it might be.
However, as the authors have run their protocol without this threshold this is easily fixed by at least briefly mentioning how this affects their results in the main text (not relegating it to the ESM where no actual discussion is provided).
From my reading of the two tables only 11 of the orders seem to have below 25\% coverage at the species-level (not 22 as the authors state in the main text!) when the threshold is used, and this improves slightly to 10 for when the threshold is removed.
As this difference isn't dramatic it seems worth mentioning.
Reviewer 2 also notes that some important studies may be missed.
This seems valid (I am aware of the biases in Ross Mounce's data collection protocol, and obviously my own) and I am sure there are many more data sets that could be sampled.
In particular I am aware of a number of studies that can only be found in books and hence are not easily accessed even if they occur in a google scholar search.
I do not expect the authors to more thoroughly search the literature for all possible cladistic analyses of mammals, however, as this is a major task.
(I should know, it has been a long term project of mine for years.)
Nevertheless, it does deserve mention in the discussion as a potential bias in the results.
I don't think this undermines the study, but rather should be viewed as both a reason for optimism and an acknowledgement of a separate problem (i.e., access to cladistic data) which multiple publications already cover (e.g., see Ross Mounce's work).}}

We thank the reviewer for their practical stance on recollecting the dataset. We have added some comments to the manuscript to deal with some of the biases in our procedure and why some matirces will be missing.
We also removed the character threshold and specified what we meant by ``data accessibility'' in the discussion (see Reviewer 2 comment 2). See also our responses to the reviewer comments above.

\item{\textcolor{blue}{Reviewer 3 (Pete Wagner)'s major problem - with which I agree - is semantic, and thus easily fixable with a few explicit caveats (e.g., ``by cladistic we mean...'').}}

We modified the mention of ``cladistic characters'' to ``anatomical characters'' throughout the manuscript (see reviewer 3 comment 1).

\item{\textcolor{blue}{One point I would add of my own is that the orders used are obviously very variable in size (species number), so what about total coverage?
This could surely be synthesised from the tables and would be a nice simple way to compare the with and without threshold results.
Similarly, I think Reviewer 3's [We assume Graeme Lloyd is refering to Graham Slater's reviewe (reviewer 2)] %or same as comment before: no names!
 graphs are a nice potential addition to the ESM.}}

We added the data coverage among all mammals in the Table 1 (``Mammalia (Class)'') and added it to the methods, results and discussion.
We also added the Reviewer 2's extra analysis suggestion to the ESM 2 part 1.

\end{enumerate}

\subsection{Reviewer 4 specific comments:}
\begin{enumerate}
\item{\textcolor{blue}{Lines 72-74 - Web addresses should be consistent in dropping of ``http://www.''.}}
We removed the ``http://www.'' part of the URLs.

\item{\textcolor{blue}{Line 104 - ``NTI 15'' should probably be ``NTI [15]''.}}
%TO DO IN THE WORD MANUSCRIPT! 
We fixed this typo.

\item{\textcolor{blue}{Line 104 - It's not completely clear to me how ``distance'' is measured. Is this branch-lengths on a time-scaled tree? Or just a node count?}}
We have clarified this by changing ``observed mean distance between each of $n$ taxa'' to ``observed mean sum of the branch lengths between each of $n$ taxa''.

\item{\textcolor{blue}{Line 117 - It's not clear to me how supraspecific values were calculated if the reference phylogeny (ref 10) is species-level. Are you using the node subtending the higher taxon or something else?}}
We pruned Bininda-Emonds et al's phylogeny for each order in order to include only the species within this order using the the functions:

\texttt{nodeX <- ape::getMRCA(treeAllMammals, listOfSpeciesFromOrderX)}\\
\texttt{treeOrderX <- ape::drop.tip(treeAllMammals, nodeX)}\\

We added the following sentence to in the methods:
``For each analysis our focal taxa were those with available anatomical characters at that taxonomic level and the phylogeny was that of the order pruned from [10]. For each order, we pruned the phylogeny from [10] to include only the species/genus/families from this order.''. p@. lines@@.


\item{\textcolor{blue}{Line 149 - ``most palaeontological studies use the genus as their smallest OTU'' I'm not sure if this is strictly true. It's more that shorter OTU names are commonly used. After all at least one actual specimen has to be coded and in my experience if you read the paper usually only a single species is used. Palaeontologists are just bad at using proper binomials in their matrices (which is an issue, but a different one). In addition ref 20 is actually all species-level OTUs! (I didn't check ref 21.)}}

We modified this statement by stating that ``many palaeontological studies use the genus (or monospecific genera) as their smallest OTU''.

\item{\textcolor{blue}{Line 154 - ``phylogenetically overdispersed'' Purely semantic, but I guess what you mean is that even sampling would be ideal, and this is *measured* using phylogenetic over-dispersion.}}

We changed ``the ideal scenario is for them to be phylogenetically overdispersed'' to ``the ideal scenario is for them to be evenly distributed (as measured by phylogenetic overdispersion)''.

\item{\textcolor{blue}{Line 245 - The caption could be more informative regarding the random and clustered distributions that these figures represent.}}

We changed the caption to: ``Phylogenetic distribution of species with available cladistic data across two orders (A: Primates; B: Carnivora). Blue branches indicate available cladistic data for the species. Available cladistic data is randomly distributed in Primates (A) and clustered in Carnivora (B).''.

\item{\textcolor{blue}{ESM 1 - Incomplete sentence: ``The list of all the 286 download matrices is available on''}}

We fixed this typo. See reviewer 1 specific comment 12.

\item{\textcolor{blue}{ESM 1 - Incomplete sentence: ``All the standardised matrices are available on''}}

We fixed this typo. See reviewer 1 specific comment 12.

\item{\textcolor{blue}{ESM 1 - ``designated as ``fossil'' all OTUs that were present in the Paleobiology database'' The database contains extant taxa too, although these should be marked as such. I guess this doesn't matter given the flow chart in Figure 2, but is something the authors should be aware of. (Obviously more generally living species can have fossil records, or at least plenty of fossils are assigned to extant species.)}}

We do agree with this point (in a similar way: \textit{Thylacinus cynocephalus} is present in Wilson Reader's list but can not be designated as ``living'') and thus added a precision to what we mean by ``fossil'' OTU in the ESM 1: ``i.e. OTUs not present in the two previous data sets [Wilson Reeder and Binida-Emonds]''.

\item{\textcolor{blue}{ESM 1 Figure 2 - How many OTUs were ignored? And based on the (assumed) cursory inspection of the authors what are they? Higher taxa? Specimens? Or just species you can't reconcile with your three data sets?}}

1549 of the 11010 OTUs were ignored.
We added the following sentence to the ESM 1 to explain which OTUs were considered as NAs.
``Non-applicable OTUs were either specimen IDs with no related taxa (e.g. \textit{FMNHPR2081}), abbreviation that were not described in the associated paper (e.g. \textit{Ho.sap.}), non-mammals \textit{stricto-sensu} (e.g. \textit{Sinoconodon}), non standard taxonomic level (e.g. \textit{Spalcotheriids}) invalid taxonomic designation (e.g. \textit{sp\_nov\_1} or \textit{Outgroup}) or typos (e.g. \textit{Hobo sabpiens}).''

\item{\textcolor{blue}{ESM 2 - ``The following section contains supplementary results to the main body: the available data structure using the NTI and the PD metric'' Abbreviations should be introduced when first used, not later as they are here.}}
We changed this to ``using the Net Relatedness Index (NRI) and the Nearest Taxon Index (NTI) metrics''.

\end{enumerate}



We hope we have responded to all these comments appropriately. Please let us know if you require any further information,\\
\bigskip

% Editorial office comments to authors:
% Please ensure that you include; 
% *An author contributions section which fulfils all *four* of our criteria given here https://royalsociety.org/journals/ethics-policies/openness/, including a sentence stating all authors agree to be held accountable for the content therein and approve the final version of the manuscript 
%TG: Ok, we I'm not sure here. Is it just an automatic message: did all the author contributed well enough to be author (yes!); or is it that we have to modify the author contribution list to the exact wordings they put in this URL (seems odd though).

% *A funding statement which is the same as that given in ScholarOne
%TG: Yes, is matching exactly.

% *Please also confirm in your cover letter whether all the figures are your own or whether permission has been obtained for their use
%TG: TO DO (yes)
% *Please upload your original figure files as eps, tiff or jpeg files, rather than PDFs
%TG: TO DO (easy)
% *If you have any images that can be used to promote your article on social media (should this be accepted) please upload them as a supplementary file
%TG: what about the "phylogeny" with the duck and the pig?
% *Please confirm whether your manuscript should be available via open access, as we note that this may be a requirement of your funder.
%TG: TO DO (yes)

%TG: just fix/add all this in this document
%TG: Make sure word limit is kept to 2500 +/- 5%

Thomas Guillerme (t.guillerme@imperial.ac.uk)\\ % Changing email + adding imperial address? % NC: Just change email. For the paper, you can change your corresponding address but I wouldn't add imperial as another proper address as you did most of the work at trinity.
Natalie Cooper

\end{document}