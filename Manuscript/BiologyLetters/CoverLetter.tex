\documentclass[11pt]{letter}
\usepackage[a4paper,left=2.5cm, right=2.5cm, top=1cm, bottom=1cm]{geometry}
\usepackage[osf]{mathpazo}
\usepackage{url}
\signature{Thomas Guillerme \\ Natalie Cooper}
\address{School of Natural Sciences \\ Trinity College Dublin \\ Dublin 2, Ireland \\ guillert@tcd.ie}
\longindentation=0pt
\begin{document}

\begin{letter}{}
\opening{Dear Editors,}

%TG: That's word for word the TEM paper cover letter
In recent years there has been growing interest in building phylogenies that contain both living and fossil taxa (e.g. Quental and Marshall 2006 TREE; Fritz et al. 2013 TREE; Heath et al. 2014 PNAS).
Such phylogenies could revolutionize the way we think about macroevolutionary patterns and processes, and provide a more complete understanding of trends in biodiversity through time.

One method, the Total Evidence method, allows us to use molecular and morphological data to build phylogenies with both living and fossil species as tips (Ronquist et al. 2012 Syst Biol).
%TG: it changes from here:
This method is extremely promising because it allows us to use all the available data. However, to efficiently apply this method, it is crucial to have enough morphological data for living species for allowing the fossil ones to branch correctly into the phylogeny (Guillerme and Cooper 2016, Molecular Phylogenetics and Evolution).

Our research article, entitled ``Assessment of cladistic data availability for living mammals'', is to our knowledge, the first to thoroughly assess the availability of cladistic (morphological) data in living mammals.
Additionally, we test whether the available data in each mammalian order is randomly distributed or clustered towards certain clades to investigate whether the placement of fossil taxa within these order would have respectively no \textit{a priori} bias or would be biased towards the clades with the most data.

Our results show that 22 of 28 mammalian orders have less than 25\% species with available morphological data.
Fortunately, however, most of this available data was randomly distributed across each mammalian order expect for 6 for which the data was biased towards certain clades.
We hope that our work will encourage cladisticians %TG: or is that how there are called?
to sample more data for living species and thus serving as a stepping stone for improving combining living and fossil taxa in the same phylogenies.

We look forward to hearing from you soon,


\closing{Yours sincerely,}


\end{letter}
\end{document}

% Editors
% Paul Barrett? Or do they have one assigned for the issue already

% Suggested reviewers:
% Fred Ronquist
% Alex Pyron (since he already reviewed the other one)
% Phil Donoghue (but conflict of interest?)
% O'Leary (even if you know I hate the Science paper, we mention morphoBank in a positive light)

