\documentclass[11pt]{letter}
\usepackage[a4paper,left=2.5cm, right=2.5cm, top=1cm, bottom=1cm]{geometry}
\usepackage[osf]{mathpazo}
\usepackage{url}
\signature{Thomas Guillerme \\ Natalie Cooper}
\address{School of Natural Sciences \\ Trinity College Dublin \\ Dublin 2, Ireland \\ guillert@tcd.ie}
\longindentation=0pt
\begin{document}



% PLoS Biology cover letter:
% Upload a cover letter as a separate file in the online system. The length limit is 600 words.

% The cover letter should address the following questions:

%     What is the scientific question you are addressing?
% "investigate whether mammal diversity was directly affected by the K-Pg mass extinction."
%     What is the key finding that answers this question?
% "We find no evidence for a direct effect of the K-Pg extinction event on mammalian diversification."
%     What is the nature of the evidence you provide in support of your conclusion?
% The time slicing method and using both data?
%     What are the three most recently published articles that are relevant to this question?
% Slater 2013 Methods Ecol. Evol.; Beck \& Lee 2014 Proc. Roy. Soc. B Close et al. 2015 Current Biol. (Or maybe put O'Leary?)
%     What significance do your results have for the field?
% Proposing new methods to investigate disparity-through-time in a continuous way + using both living and fossil taxa
%     What significance do your results have for the broader community (of biologists and/or the public)?
% Text book example.
%     What other novel findings do you present?
% Effect of using both data sources
%     Is there additional information that we should take into account?
% Graeme's story?


% TG: Ok, nothing too complex here, just some reformatting maybe. Everything is clearly stated in the Nature Comm version but not in their order. Maybe do a bullet point alternative version?

\begin{letter}{}
\opening{Dear Editors,}

Combining fossils into phylogenies is cool
We can use TEM method along side with tip-dating
However, to eficiently do it, it is important to have enough data for living species for allowing the fossil ones to branch correctly (Guillerme and Cooper 2016, Molecular Phylogenetics and Evolution)

In this research article, entitled ``Investigating the effects of the Cretaceous-Paleogene mass extinction on mammalian morphological diversity using a new methodological approach'', we use both data from living and fossil species through Total Evidence tip-dated phylogenies along with a novel continuous time-slicing method to investigate whether mammalian morphological diversity was directly affected by the K-Pg event.

We look forward to hearing from you,

\closing{Yours sincerely,}


\end{letter}
\end{document}

% Editors
% Dean Adams?
% Matt Friedman?
% Daniel Rabosky?
% Liam Revell?

% Suggested reviewers:
% Graeme Lloyd
% Robin Beck
% Robert Close
% Mike Benton or Steve Brusatte or Richard Butler (they're cited a lot)
% Any other ideas?

% Yep all good.
% BUT if you want to suggest Graeme you need to add the postscript I added above. You may need to alter the reference.


