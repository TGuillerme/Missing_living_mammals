\documentclass[11pt]{letter}
\usepackage[a4paper,left=2.5cm, right=2.5cm, top=1cm, bottom=1cm]{geometry}
\usepackage[osf]{mathpazo}
\usepackage{url}
\signature{Thomas Guillerme \\ Natalie Cooper}
\address{School of Natural Sciences \\ Trinity College Dublin \\ Dublin 2, Ireland \\ guillert@tcd.ie}
\longindentation=0pt
\begin{document}

\begin{letter}{}
\opening{Dear Editors,}

Please find attached our research article, ``Assessment of cladistic data availability for living mammals'', for consideration in the Biology Letters Special Issue: ``Putting fossils into trees''.% NC: What is the title?
%TG: TO CHECK

% NC: For special issues you need to make this clear upfront so it doesn't go through the normal channels!

%TG: That's word for word the TEM paper cover letter
In recent years there has been growing interest in building phylogenies that contain both living and fossil taxa (e.g. Quental and Marshall 2006 TREE; Fritz et al. 2013 TREE; Heath et al. 2014 PNAS).
Such phylogenies could revolutionize the way we think about macroevolutionary patterns and processes, and provide a more complete understanding of trends in biodiversity through time.

The Total Evidence method allows us to use molecular and morphological data to build phylogenies with both living and fossil species as tips (Ronquist et al. 2012 Syst Biol).
%TG: it changes from here:
This method is extremely promising because it allows us to use all the available data. 
However, to efficiently apply this method, it is crucial to have enough morphological data for living species to allow the fossils to branch correctly in the phylogeny (Guillerme and Cooper 2016, Molecular Phylogenetics and Evolution).

Our research article assesses the availability of coded cladistic (morphological) data in living mammals within existing databases and data compilations.
Additionally, we test whether the available data in each mammalian order is randomly distributed or clustered towards certain clades to investigate whether the placement of fossil taxa within these orders would be biased towards the clades with the most data.

Our results show that 22 of 28 mammalian orders have \textless 25\% species with available cladistic data.
Fortunately, however, most of this available data was randomly distributed across each mammalian order except for six orders where the data was biased towards certain clades.
We hope that our work will encourage researchers to collect more data for living species and thus help to improve attempts to combine living and fossil taxa in the same phylogenies.

We look forward to hearing from you soon,


\closing{Yours sincerely,}


\end{letter}
\end{document}

% Editors
% Paul Barrett? Or do they have one assigned for the issue already
% NC: Put Paul down if it asks you to. Usually these are edited by the organisers so April, Nick and Graeme.

% Suggested reviewers:
% Fred Ronquist
% Alex Pyron (since he already reviewed the other one)
% Phil Donoghue (but conflict of interest?)
% O'Leary (even if you know I hate the Science paper, we mention morphoBank in a positive light)

% NC: They seem ok, I think you can put Phil. He isn't working with you on anything and he is not your friend as such. He has nothing to gain from you getting this published (I'd have no issues reviewing for Kevin Arbuckle). If you want others think of museum people doing morphological phylogenies, maybe in the US? Not sure about O'Leary because she is a pain apparently.
