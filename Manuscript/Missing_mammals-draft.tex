%Authors guidlines: http://royalsocietypublishing.org/instructions-authors#question5

\documentclass[12pt,letterpaper]{article}

%Packages
\usepackage{pdflscape}
\usepackage{fixltx2e}
\usepackage{textcomp}
\usepackage{fullpage}
\usepackage{natbib}
\usepackage{float}
\usepackage{latexsym}
\usepackage{url}
\usepackage{epsfig}
\usepackage{graphicx}
\usepackage{amssymb}
\usepackage{amsmath}
\usepackage{bm}
\usepackage{array}
\usepackage[version=3]{mhchem}
\usepackage{ifthen}
\usepackage{caption}
\usepackage{hyperref}
\usepackage{amsthm}
\usepackage{amstext}
\usepackage{enumerate}
\usepackage[osf]{mathpazo}
\usepackage{dcolumn}
\usepackage{lineno}
\pagenumbering{arabic}


%Pagination style and stuff
\linespread{2}
\raggedright
\setlength{\parindent}{0.5in}
\setcounter{secnumdepth}{0} 
\renewcommand{\section}[1]{%
\bigskip
\begin{center}
\begin{Large}
\normalfont\scshape #1
\medskip
\end{Large}
\end{center}}
\renewcommand{\subsection}[1]{%
\bigskip
\begin{center}
\begin{large}
\normalfont\itshape #1
\end{large}
\end{center}}
\renewcommand{\subsubsection}[1]{%
\vspace{2ex}
\noindent
\textit{#1.}---}
\renewcommand{\tableofcontents}{}
\bibpunct{(}{)}{;}{a}{}{,}

%---------------------------------------------
%
%       START
%
%---------------------------------------------

\begin{document}

%Running head
\begin{flushright}
Version dated: \today
\end{flushright}
\bigskip
\noindent RH: Missing morphological data in living mammals

\bigskip
\medskip
\begin{center}

\noindent{\Large \bf Missing morphological data in living mammals}

\bigskip

\noindent {\normalsize \sc Thomas Guillerme$^1$$^,$$^2$$^*$, and Natalie Cooper$^1$$^,$$^2$}\\
\noindent {\small \it 
$^1$School of Natural Sciences, Trinity College Dublin, Dublin 2, Ireland.\\
$^2$Trinity Centre for Biodiversity Research, Trinity College Dublin, Dublin 2, Ireland.}\\
\end{center}
\medskip
\noindent{*\bf Corresponding author.} \textit{Zoology Building, Trinity College Dublin, Dublin 2, Ireland; E-mail: guillert@tcd.ie; Fax: +353 1 6778094; Tel: +353 1 896 2571.}\\
\vspace{1in}

%Line numbering
\modulolinenumbers[1]
\linenumbers

%---------------------------------------------
%
%       ABSTRACT
%
%---------------------------------------------

\newpage
\begin{abstract}

\end{abstract}

\noindent ()\\

\vspace{1.5in}

\newpage 

%---------------------------------------------
%
%       INTRODUCTION
%
%---------------------------------------------

\section{Introduction}
%Restrict to two paragraphs?

We need trees with both living and fossil taxa blablablablabla

However, because of the nature of TEM matrices, we can have a lot of missing data blabalbalbal

Some missing has more effect on recovering the correct topology than others. Missing living taxa are important! Or they are available (because we have molecular data at least).

In this study we want to have a accurate overlook of the state of morphological availability in mammals.

Also, when data is available, we want to know how the data is distributed along the clade. Because it is irrealistic or unnecessary to sample the whole clade, we want the data to be at least randomly distributed (so no bias) or at the best evenly distributed (so every clade is sampled). However, what we don't want is that data to be clustered.

After looking at the data availability and it's distribution, we propose several strategies to improve data coverage in weekly sampled orders.

Questions:
\begin{enumerate}
\item{-How many taxa with morphological data are available among each living mammals order?}
\item{-Within each order, how is this data distributed?}
\item{-How can we improve the data coverage in taxa with non random distributed data}
\end{enumerate}


%---------------------------------------------
%
%       METHODS
%
%---------------------------------------------
 
\newpage

\section{Material and Methods}
\subsection{Matrices search}
To investigate the available living taxa with morphological data, we downloaded morphological matrices from three main public databases: morphobank, graeamlloyd.com and rossmounce's github. We downloaded all the matrices containing any fossil or living mammal taxa from these data bases. Additionally we ran a thorough search for matrices that might not have been uploaded on the previously cited data bases through a Google Scholar search. We downloaded the eventual additional morphological matrices from any of the 20 first papers matching with our selected key words and with any of the 35 taxonomic levels (see supplementary materials for detailed description of the procedure). We downloaded @@@ % Number of matrices
matrices containing a total of @@@%Number of living OTUs
operational taxonomic units (OTUs) from the combination of both searches (public repositories and Google Scholar).

We then transformed all the matrices to be in the same nexus format. We then standardised the taxonomic nomenclature by fixing invalid binomial inputs to match with the official taxonomic nomenclature rules (i.e. H. sapiens was transformed in Homo sapiens). We assigned each species as being either living or fossil using a taxonomic matching algorithm. We considered living all the OTUs that where either present in Fritz supper tree or in Wilson Reeders taxonomy. We considered fossil all the OTUs that where present in the Paleobiology database. For the OTUs neither labelled as living or fossil we tried to decompose the taxa name (i.e. Homo_sapiens became Homo and sapiens) and tried to match the to Wilson Reeders taxonomy at any taxonomic level (Genus_species, Genus, Family, etc...). The matching taxa where labelled as living and the ones still not matching where ignored and labelled as not applicable (NA).

\subsection{Data availability analysis}
\subsubsection{Data availability}
Relative proportion of available data per order.
\subsubsection{Spread of the available data}
PD/NRI/NTI to null.
\subsubsection{Data spread improvement strategy}
Phylotargeting

All the following procedure is repeatable and available on github.

%---------------------------------------------
%
%       RESULTS
%
%---------------------------------------------

\section{Results}
Table 1: data availability

Table 2: randomness

Table 3: phylotargeting

Figure 1: example of good and bad coverage (artiodactyles and carnivores)?

%---------------------------------------------
%
%       DISCUSSION
%
%---------------------------------------------

\section{Discussion}

How to compare morphological characters?

How to combine these matrices?

%Biology letters various stuff
\section{Ethics statement}
\section{Data accessibility statement}
All data is available and reproducible on github.
\section{Authors’ contributions statement}
Conceived and designed the experiments: TG NC. Performed the experiments: TG. Analyzed the data: TG. Wrote the paper: TG NC.
\section{Acknowledgements}
April Wright, David Bapst and Graeme Lloyd.
\section{Funding statement}
This work was funded by a European Commission CORDIS Seventh Framework Programme (FP7) Marie Curie CIG grant (proposal number: 321696).



%\bibliographystyle{}
%\bibliography{}


%Tables
%Figures


\section{SOM}
1- Data collection: key words, clade (ordinal) metacharacters, Google Search terms, Google Search protocol, Google Search rarefaction curve.

\section{search terms}

\subsection{Mammalian orders terms}
The searched mammalian order terms are available in search_terms_latin.txt or search_terms_meta.txt. The file containing the meta names is the Latin name files but with replacing the latin suffixes ([ia|ata|ea|a]) by a joker character (*) and by replacing the first letter by a upper/lower case meta character (e.g. [Aa]).

\subsubsection{Ross Mounce data set}
I selected all the matrices containing at least one of the mammalian orders names from Ross Mounce GitHub cladistic-data/nexus_files repository (accessed on the 02/12/2014).

\subsection{Graeme Lloyd}
Selecting the downloadable matrices

\subsection{Morphobank}
\textit{order}

\subsection{Google scholars}
20 first results since 2010 with the following key words:
\textit{order} ("morphology" OR "morphological" OR "cladistic") AND characters matrix paleontology phylogeny
Why 20 first results? Because rarefaction curve, check supplemetaries

\section{Wrong Bionmial names and typos}
I fixed the wrong bionomial names format (e.g. H. sapiens) into the correct ones (e.g. Homo sapiens) manually using the abreviation list in the concerned publications.


%END
\end{document}