%Authors guidlines: http://royalsocietypublishing.org/instructions-authors
% 2500 words max (includes the title page, abstract, references, acknowledgements and figure/table legends)
% current version is around 3700. I think a big cut down can be done on the references.
% We allow a maximum of 4 displays, only 2 of which can be figures.

\documentclass[12pt,letterpaper]{article}
\usepackage{natbib}

%Packages
\usepackage{pdflscape}
\usepackage{fixltx2e}
\usepackage{textcomp}
\usepackage{fullpage}
\usepackage{float}
\usepackage{latexsym}
\usepackage{url}
\usepackage{epsfig}
\usepackage{graphicx}
\usepackage{amssymb}
\usepackage{amsmath}
\usepackage{bm}
\usepackage{array}
\usepackage[version=3]{mhchem}
\usepackage{ifthen}
\usepackage{caption}
\usepackage{hyperref}
\usepackage{amsthm}
\usepackage{amstext}
\usepackage{enumerate}
\usepackage[osf]{mathpazo}
\usepackage{dcolumn}
\usepackage{lineno}
\usepackage{longtable}
\pagenumbering{arabic}

\newcolumntype{L}[1]{>{\raggedright\let\newline\\\arraybackslash\hspace{0pt}}m{#1}}
\newcolumntype{C}[1]{>{\centering\let\newline\\\arraybackslash\hspace{0pt}}m{#1}}
\newcolumntype{R}[1]{>{\raggedleft\let\newline\\\arraybackslash\hspace{0pt}}m{#1}}

%Pagination style and stuff % NC: Note that these are all syst biol specific.
\linespread{2}
\raggedright
\setlength{\parindent}{0.5in}
\setcounter{secnumdepth}{0} 
\renewcommand{\section}[1]{%
\bigskip
\begin{center}
\begin{Large}
\normalfont\scshape #1
\medskip
\end{Large}
\end{center}}
\renewcommand{\subsection}[1]{%
\bigskip
\begin{center}
\begin{large}
\normalfont\itshape #1
\end{large}
\end{center}}
\renewcommand{\subsubsection}[1]{%
\vspace{2ex}
\noindent
\textit{#1.}---}
\renewcommand{\tableofcontents}{}
%\bibpunct{(}{)}{;}{a}{}{,}

%---------------------------------------------
%
%       START
%
%---------------------------------------------

\begin{document}

%Running head
\begin{flushright}
Version dated: \today
\end{flushright}
\bigskip
\noindent RH: Morphological data availability in living mammals

\bigskip
\medskip
\begin{center}

\noindent{\Large \bf Morphological data is lacking for living mammals}
%Or: Missing cladistic data in living mammals - but I prefer the first one emphasizing on the availability of it. The data is not missing per se, it's just not coded!
% NC: sometimes the better title is one that answers the question rather than giving the topic. e.g., "Morphological data is lacking for living species" or similar
% TG: maybe worth considering the special issue as well? "Putting fossils in tree" so maybe something like "There is not enough morphological data for living mammals to branch fossils in the trees."
% TG: or some poetic... "Living and dead mammals: morphological data is lacking in living for combined analysis"

%Key words: Total Evidence method, data structure, phylogenetic, living, fossil, topology
\bigskip

\noindent {\normalsize \sc Thomas Guillerme$^1$$^,$$^2$$^,$$^*$ and Natalie Cooper$^1$$^,$$^2$$^,$$^3$}\\
\noindent {\small \it 
$^1$School of Natural Sciences, Trinity College Dublin, Dublin 2, Ireland.\\
$^2$Trinity Centre for Biodiversity Research, Trinity College Dublin, Dublin 2, Ireland.\\
$^3$Department of Life Sciences, Natural History Museum, Cromwell Road, London, SW7 5BD, UK.}\\
\end{center}
\medskip
\noindent{*\bf Corresponding author.} \textit{Zoology Building, Trinity College Dublin, Dublin 2, Ireland; E-mail: guillert@tcd.ie; Fax: +353 1 6778094; Tel: +353 1 896 2571.}\\  
\vspace{1in}

%Line numbering
\modulolinenumbers[1]
\linenumbers

%---------------------------------------------
%
%       ABSTRACT
%
%---------------------------------------------
% NC: I've commented this out so I don't have to keep scrolling past in in the PDF
%\newpage
%\begin{abstract} % 200 words, all keywords written in abstract if possible.
%Studying changes in global biodiversity through time and space is essential.
%For that we need methods to combine both palaeontological and neontological data.
%One promising method, the Total Evidence method, allows such thing but needs a lot of data.
%Especially cladistic data from living taxa to allow the fossil taxa to accurately branch in the trees.
%Despite two centuries of morphological studies on living taxa, scientists using and generating such data mainly focus on palaeontological data.
%Therefore, even in well known groups such as mammal, there is a huge gap in our knowledge of cladistic data for living mammals.
%In this study, using phylogenetic community structure methods, we quantify the availability of data in each mammalian order.
%And maybe at the end of the paper we propose some discussion on how to improve all that (go in museums!).
%\end{abstract}

%\noindent ()\\

%\vspace{1.5in}

%---------------------------------------------
%
%       INTRODUCTION
%
%---------------------------------------------
\newpage 
\section{Introduction}
There is an increasing consensus among evolutionary biologists that studying both living and fossil taxa is essential for fully understanding macroevolutionary patterns and processes \citep{jacksonwhat2006,quentaldiversity2010,dietlconservation2011,slaterunifying2013,fritzdiversity2013,Wood01032013}.
For example, including both living and fossil taxa in evolutionary studies can improve the accuracy of timing diversification events \citep[e.g.][]{ronquista2012}, our understanding of relationships among lineages \citep[e.g.][]{beckancient2014}, and our ability to infer biogeographical patterns through time \citep[e.g.][]{Meseguer01032015}.
To perform such analyses it is necessary to combine living and fossil taxa in phylogenetic trees.
One increasingly popular method, the Total Evidence method \citep{eernissetaxonomic1993,ronquista2012}, combines molecular data from living taxa and morphological data from both living and fossil taxa in a supermatrix \citep[e.g.][]{pyrondivergence2011,ronquista2012,schragocombining2013,slaterunifying2013,beckancient2014,Meseguer01032015}, producing a phylogeny with living and fossil taxa at the tips. 
These phylogenies can be dated using methods such as tip-dating \citep{ronquista2012,Drummond01082012,Wood01032013} and incorporated into macroevolutionary studies \citep[e.g.][]{ronquista2012,Wood01032013,slaterphylogenetic2013}.

A downside of the Total Evidence method is that it requires a lot of data.
One must collect molecular data for living taxa and morphological data for both living and fossil taxa; two types of data that require fairly different technical skills (e.g. molecular sequencing \textit{vs.} anatomical description).
Additionally, large chunks of this data can be difficult, or even impossible, to collect for every taxon present in the analysis.
For example, fossils very rarely have molecular data and incomplete fossil preservation (e.g. soft \textit{vs.} hard tissues) may restrict the amount of morphological data available \citep{sansomfossilization2013}.
Additionally, since the molecular phylogenetics revolution, it has become less common to collect morphological characters for living taxa when molecular data is available (e.g. in \cite{slaterphylogenetic2013}, only 13\% of the 169 living taxa have coded morphological data).
Unfortunately this missing data can lead to errors in phylogenetic inference; in fact, simulations show that the ability of the Total Evidence method to recover the correct phylogenetic topology decreases when there is a low overlap between morphological data in the living and fossil taxa \citep{GuillermeCooper}, regardless the overall amount of morphological data available for the fossils (or the amount of molecular data available for the living species).
The effect of missing data on topology is greatest when living taxa have few morphological data.
This is because (1) a fossil cannot branch in the correct clade if there is no overlapping morphological data in the clade; and (2) a fossil has a higher probability of branching within a clade with more morphological data available for living taxa, regardless of whether this is the correct clade or not \citep{GuillermeCooper}. 

The issues above highlight that it is crucial to have sufficient morphological data for living taxa in a clade before using a Total Evidence approach.
However, it is unclear how much morphological data for living taxa is actually available (i.e. already coded from museum specimens and deposited in phylogenetic matrices accessible online), and how this data is distributed across clades.
Intuitively, most people assume this kind of data has already been collected, but empirical data suggest otherwise (e.g. in \cite{ronquista2012,slaterphylogenetic2013,beckancient2014}).
To investigate this further, we assess the amount of available morphological data for living mammals to determine whether sufficient data exists to build reliable Total Evidence phylogenies in this group.
We collected cladistic data (i.e. discrete morphological characters used in phylogenetics) from @256 phylogenetic matrices available online and measured the proportion of cladistic data available for each mammalian order.
%Additionally, if the available data is randomly distributed across clades, the topological errors should be less extreme than if the data is biased towards particulars groups.
Additionally, because missing morphological data in living species can influence tree topology as described above,%NC: Might still need to fix this sentence. "Topological caveats above" was too unclear.
we determined whether the available cladistic data was phylogenetically overdispersed or clustered in the mammalian orders where data was missing. 
We find that available morphological data for living mammals is scarce but generally randomly distributed across phylogenies. 
We recommend that efforts be made to collect and share more cladistic data for living species to improve the accuracy of Total Evidence phylogenies.

% NC: If there are word count problems we can ditch the last two sentences.
%---------------------------------------------
%
%       METHODS
%
%---------------------------------------------
\section{Materials and Methods}
\subsection{Data collection and standardisation}
We downloaded all cladistic matrices containing any living and/or fossil mammal taxa from three major public databases (accessed @th of July 2015): Morphobank \citep[\texttt{http://www.morphobank.org/};][]{morphobank}, Graeme Lloyd's website (\texttt{graemetlloyd.com/matrmamm.html}) and Ross Mounce's GitHub repository (\texttt{https://github.com/rossmounce/cladistic-data}).
We also performed a systematic Google Scholar search (accessed @th of July 2015) for matrices that were not uploaded to these databases (see Supplementary Materials for a detailed description of the search procedure).
In total, we downloaded 286 matrices containing a total of 11010 operational taxonomic units (OTUs) of which 5228 where unique. % Number of unique OTUs (including fossils)
In this study, we refer to OTUs rather than species since the entries in the downloaded matrices were not standardized and ranged from specific individual specimen (i.e. the name of a collection item) to the family-level.
Where possible, we considered OTUs at their lowest valid taxonomic level (i.e. species) but some OTUs were only valid at a higher taxonomic level (e.g. genus or family).
Therefore for some orders, we sampled more genera than species (Table \ref{Table_morpho_taxa_proportion}).

To select the lowest valid taxonomic level for each OTU, we standardised their taxonomy by correcting species names so they matched standard taxonomic nomenclature (e.g., \textit{H. sapiens} was transformed to \textit{Homo sapiens}).
We designated as ``living'' all OTUs that were either present in the phylogeny of \citet{BinindaEmonds} or the taxonomy of \citet{wilson2005mammal}, and designated as ``fossil'' all OTUs that were present in the Paleobiology database (\texttt{https://paleobiodb.org/}).
For OTUs that did not appear in these three sources, we first decomposed the name (i.e. \textit{Homo sapiens} became \textit{Homo} and \textit{sapiens}) and tried to match the first element with a higher taxonomic level (family, genus etc.).
Any OTUs that still had no matches in the sources above were designated as non-applicable (NA; see Supplementary Material for more details).

The number of characters in each matrix ranged from 6 to 4541.
Matrices with few characters are problematic when comparing available data among matrices because (1) they have less chance of having characters that overlap with those of other matrices \citep{wagner2000} and (2) they are more likely to contain a higher proportion of specific characters that are not-applicable across large clades \citep[][e.g. ``antler ramifications'' is a character that is only applicable to Cervidae not all mammals]{Brazeau2011}.
% NC: Is this true really? Does it always follow that if you've very few characters you also have very specific characters? Maybe so... TG: I don't know if it's a generalisation but from the matrices I collected, the once with few characters always came from papers focusing on intra-specific or intra-generic variations. On the other hand, I didn't read all the characters. I just assume that in a big one, you have a lower probability of having non-applicable characters (there's only some many characters for coding antler ramification)
Therefore we selected only matrices containing \textgreater 100 characters per OTU.
This threshold was chosen to correspond with the number of characters used in \citet{GuillermeCooper} and \citet{harrisonamong-character2014}.
Note that the results of the analysis with no character threshold is available in the supplementary materials.
After removing all matrices with \textless 100 characters, we retained 1074 unique living mammal OTUs from 126 matrices for our analyses. % 1601 unique living OTUs for 286 matrices (no threshold)

\subsection{Data availability and distribution}
To assess the availability of cladistic data for each mammalian order, we calculated the percentage of OTUs with cladistic data at three different taxonomic levels: family, genus and species.
We used these different taxonomic levels because some clades are well covered at the family- or genus-level, but poorly covered at the species-level.
We consider orders with \textless 25\% of living taxa with cladistic data as having poor data coverage (hereafter ``low'' coverage), and orders with \textgreater 75\% of living taxa with cladistic data as having good data coverage (hereafter ``high'' coverage). 

For orders with \textless 100\% cladistic data coverage at any taxonomic level, we investigated whether the available cladistic data was (i) randomly distributed, (ii) overdispersed or (iii) clustered, with respect to phylogeny, using two metrics from community phylogenetics: Nearest Taxon Index \citep[NTI;][]{webb2002phylogenies} and Net Relatedness Index \citep[NRI;][]{webb2002phylogenies}. 
%*** see end of methods for explanation.
Both metrics were calculated using the \texttt{picante} package in R \citep{picante,R}.
% NC: It's not going to add much to the wordcount to explain both. But move PD to supp

% NC: Note that you had mis-defined MNND
NTI \citep{webb2002phylogenies} is based on mean nearest neighbour distance (MNND) and is calculated as follows:
  \begin{equation}
    NTI=-\left(\frac{\overline{MNND}_{obs}-\overline{MNND}_{n}}{\sigma(MNND_{n})}\right)
  \end{equation}
where $\overline{MNND}_{obs}$ is the observed mean distance among $n$ focal taxa and their nearest neighbour in the phylogeny,
% NC: Sorry this still isn't quite right. Check my paper for a nicer way of writing it.
$\overline{MNND}_{n}$ is the expected random MNND for $n$ taxa estimated by calculating the MNND from $n$ taxa randomly drawn from the phylogeny and repeated 1000 times, and $\sigma(MNND_{n})$ is the standard deviation of the 1000 random MNND values. 
NRI is similar but is based on mean phylogenetic distance (MPD) as follows:
  \begin{equation}
    NRI=-\left(\frac{\overline{MPD}_{obs}-\overline{MPD}_{n}}{\sigma(MPD_{n})}\right)
  \end{equation}
where $\overline{MPD}_{obs}$ is the observed mean phylogenetic distance among $n$ focal taxa, $\overline{MPD}_{n}$ is the expected random MPD for $n$ taxa estimated by calculating the MPD from $n$ taxa randomly drawn from the phylogeny and repeated 1000 times, and $\sigma(MPD_{n})$ is the standard deviation of the 1000 random MPD values.
Negative NTI and NRI values show that the focal taxa are more overdispersed across the phylogeny than expected by chance, and positive values show that the focal taxa are more clustered across the phylogeny than expected by chance. %*** see end of methods for explanation.
We calculated NTI and NRI values for each mammalian order separately, at each different taxonomic level. 
For each analysis our focal taxa were those with available cladistic data at that taxonomic level and the phylogeny was the phylogeny of the order pruned from \citep{BinindaEmonds}. % cite  Bininda Emonds?
% NC: Tried to make this more specific here. It's important to note that you used the order not the whole mammal phylogeny each time.

%Because the aim of this study is to understand the structure of the available cladistic data in mammals we prefered the NTI over the NRI or the PD metrics.
%In fact, the NTI metric is more sensitive to the phylogenetic structure (clustering or over-dispersion) at the tips of phylogeny \citep{Cooper2008}.
%and also because previous studies have shown that NRI is biased towards detecting overdispersion \citep{Kembel2006,Swenson2006}. TG: Useless in this case?
%However, the results of the NRI and PD calculations are available in the supplementary.

% NC: Actually I think we discussed that NRI and NTI show different things? So it's worth including both (you can get the wording from my 2008 paper). This can go up with the intro to the metrics somewhere (I have *** the two places it might fit)

We also report the results using another metric, Phylogenetic Distance \citep[PD;][]{Faith19921}, in the Supplementary Information. % link?

%---------------------------------------------
%
%       RESULTS
%
%---------------------------------------------

\section{Results}
Across the @126 cladistic matrices we extracted, @22 out of 28 mammalian orders have low coverage (\textless 25\% of species with cladistic data) and @6 have high coverage (\textgreater 75\% of species with cladistic data) at the species-level.
At the genus-level, only @three order have low coverage of taxa and @12 have high coverage, and at the family-level, no orders have low coverage and @23 have high coverage (Table \ref{Table_morpho_taxa_proportion}).

% latex table generated in R 3.2.3 by xtable 1.8-0 package
% Mon Feb  8 22:09:15 2016
\begin{longtable}{lL{1.8cm}C{2cm}lcc}
\caption{Number of taxa with available cladistic data for mammalian orders at three
taxonomic levels. The left vertical bar represents low coverage (\textless 25\%; coloured in blue); the right vertical bar represents high coverage (\textgreater 75\%; coloured in orange). Negative Net Relatedness Index (NRI) and Nearest Taxon Index (NTI) values indicate phylogenetic overdispersion; positive values indicate phylogenetic clustering. Significant NRI or NTI values are in bold. *p \textless 0.05; **p \textless 0.01; ***p \textless 0.001.
} \\ 

  \hline
Order & Taxonomic level & Proportion of taxa & Coverage & NRI & NTI \\ 
  \hline
  Mammalia (class) & family & 129/148 & \includegraphics[width=0.20\linewidth, height=0.05\linewidth]{Table_figures/bar40.pdf} & -1.19 & 1.09 \\ 
  \textbf{Mammalia (class)} & \textbf{genus} & \textbf{517/1186} & \includegraphics[width=0.20\linewidth, height=0.05\linewidth]{Table_figures/bar41.pdf} & \textbf{-5.19} & \textbf{3.71**} \\ 
  \textbf{Mammalia (class)} & \textbf{species} & \textbf{847/5017} & \includegraphics[width=0.20\linewidth, height=0.05\linewidth]{Table_figures/bar42.pdf} & \textbf{-7.75} & \textbf{3.54**} \\ 
  Afrosoricida & family & 2/2 & \includegraphics[width=0.20\linewidth, height=0.05\linewidth]{Table_figures/bar1.pdf} &   &   \\ 
  Afrosoricida & genus & 17/17 & \includegraphics[width=0.20\linewidth, height=0.05\linewidth]{Table_figures/bar2.pdf} &   &   \\ 
  Afrosoricida & species & 23/42 & \includegraphics[width=0.20\linewidth, height=0.05\linewidth]{Table_figures/bar3.pdf} & 1.52 & 1.1 \\ 
  Carnivora & family & 14/15 & \includegraphics[width=0.20\linewidth, height=0.05\linewidth]{Table_figures/bar4.pdf} & 0.65 & 0.55 \\ 
  \textbf{Carnivora} & \textbf{genus} & \textbf{52/125} & \includegraphics[width=0.20\linewidth, height=0.05\linewidth]{Table_figures/bar5.pdf} & \textbf{4.27**} & \textbf{1.26} \\ 
  \textbf{Carnivora} & \textbf{species} & \textbf{75/283} & \includegraphics[width=0.20\linewidth, height=0.05\linewidth]{Table_figures/bar6.pdf} & \textbf{7.24**} & \textbf{0.8} \\ 
  Cetartiodactyla & family & 21/21 & \includegraphics[width=0.20\linewidth, height=0.05\linewidth]{Table_figures/bar7.pdf} &   &   \\ 
  Cetartiodactyla & genus & 97/128 & \includegraphics[width=0.20\linewidth, height=0.05\linewidth]{Table_figures/bar8.pdf} & 0.7 & 1.28 \\ 
  \textbf{Cetartiodactyla} & \textbf{species} & \textbf{169/310} & \includegraphics[width=0.20\linewidth, height=0.05\linewidth]{Table_figures/bar9.pdf} & \textbf{1.82*} & \textbf{-0.24} \\ 
  Chiroptera & family & 15/18 & \includegraphics[width=0.20\linewidth, height=0.05\linewidth]{Table_figures/bar10.pdf} & -0.23 & 0.61 \\ 
  \textbf{Chiroptera} & \textbf{genus} & \textbf{92/202} & \includegraphics[width=0.20\linewidth, height=0.05\linewidth]{Table_figures/bar11.pdf} & \textbf{13.07**} & \textbf{0.99} \\ 
  \textbf{Chiroptera} & \textbf{species} & \textbf{214/1053} & \includegraphics[width=0.20\linewidth, height=0.05\linewidth]{Table_figures/bar12.pdf} & \textbf{9.21**} & \textbf{1.27} \\ 
  Cingulata & family & 1/1 & \includegraphics[width=0.20\linewidth, height=0.05\linewidth]{Table_figures/bar13.pdf} &   &   \\ 
  Cingulata & genus & 8/9 & \includegraphics[width=0.20\linewidth, height=0.05\linewidth]{Table_figures/bar14.pdf} & 1.48 & -1.54 \\ 
  \textbf{Cingulata} & \textbf{species} & \textbf{9/29} & \includegraphics[width=0.20\linewidth, height=0.05\linewidth]{Table_figures/bar15.pdf} & \textbf{2.06*} & \textbf{0.2} \\ 
  Dasyuromorphia & family & 2/2 & \includegraphics[width=0.20\linewidth, height=0.05\linewidth]{Table_figures/bar16.pdf} &   &   \\ 
  Dasyuromorphia & genus & 8/22 & \includegraphics[width=0.20\linewidth, height=0.05\linewidth]{Table_figures/bar17.pdf} & -0.78 & -1.06 \\ 
  Dasyuromorphia & species & 9/64 & \includegraphics[width=0.20\linewidth, height=0.05\linewidth]{Table_figures/bar18.pdf} & -0.86 & -0.37 \\ 
  Dermoptera & family & 1/1 & \includegraphics[width=0.20\linewidth, height=0.05\linewidth]{Table_figures/bar19.pdf} &   &   \\ 
  Dermoptera & genus & 1/2 & \includegraphics[width=0.20\linewidth, height=0.05\linewidth]{Table_figures/bar20.pdf} &   &   \\ 
  Dermoptera & species & 1/2 & \includegraphics[width=0.20\linewidth, height=0.05\linewidth]{Table_figures/bar21.pdf} &   &   \\ 
  Didelphimorphia & family & 1/1 & \includegraphics[width=0.20\linewidth, height=0.05\linewidth]{Table_figures/bar22.pdf} &   &   \\ 
  Didelphimorphia & genus & 16/16 & \includegraphics[width=0.20\linewidth, height=0.05\linewidth]{Table_figures/bar23.pdf} &   &   \\ 
  Didelphimorphia & species & 42/84 & \includegraphics[width=0.20\linewidth, height=0.05\linewidth]{Table_figures/bar24.pdf} & -1.61 & 0.12 \\ 
  Diprotodontia & family & 11/11 & \includegraphics[width=0.20\linewidth, height=0.05\linewidth]{Table_figures/bar25.pdf} &   &   \\ 
  Diprotodontia & genus & 25/38 & \includegraphics[width=0.20\linewidth, height=0.05\linewidth]{Table_figures/bar26.pdf} & -1.15 & -1.33 \\ 
  Diprotodontia & species & 31/126 & \includegraphics[width=0.20\linewidth, height=0.05\linewidth]{Table_figures/bar27.pdf} & 0.44 & -1.79 \\ 
  Erinaceomorpha & family & 1/1 & \includegraphics[width=0.20\linewidth, height=0.05\linewidth]{Table_figures/bar28.pdf} &   &   \\ 
  Erinaceomorpha & genus & 10/10 & \includegraphics[width=0.20\linewidth, height=0.05\linewidth]{Table_figures/bar29.pdf} &   &   \\ 
  Erinaceomorpha & species & 21/22 & \includegraphics[width=0.20\linewidth, height=0.05\linewidth]{Table_figures/bar30.pdf} & -1.04 & -0.25 \\ 
  Hyracoidea & family & 1/1 & \includegraphics[width=0.20\linewidth, height=0.05\linewidth]{Table_figures/bar31.pdf} &   &   \\ 
  Hyracoidea & genus & 1/3 & \includegraphics[width=0.20\linewidth, height=0.05\linewidth]{Table_figures/bar32.pdf} &   &   \\ 
  Hyracoidea & species & 1/4 & \includegraphics[width=0.20\linewidth, height=0.05\linewidth]{Table_figures/bar33.pdf} &   &   \\ 
  Lagomorpha & family & 2/2 & \includegraphics[width=0.20\linewidth, height=0.05\linewidth]{Table_figures/bar34.pdf} &   &   \\ 
  Lagomorpha & genus & 5/12 & \includegraphics[width=0.20\linewidth, height=0.05\linewidth]{Table_figures/bar35.pdf} & -0.95 & -0.94 \\ 
  Lagomorpha & species & 12/86 & \includegraphics[width=0.20\linewidth, height=0.05\linewidth]{Table_figures/bar36.pdf} & -0.62 & -1.96 \\ 
  Macroscelidea & family & 1/1 & \includegraphics[width=0.20\linewidth, height=0.05\linewidth]{Table_figures/bar37.pdf} &   &   \\ 
  Macroscelidea & genus & 4/4 & \includegraphics[width=0.20\linewidth, height=0.05\linewidth]{Table_figures/bar38.pdf} &   &   \\ 
  Macroscelidea & species & 12/15 & \includegraphics[width=0.20\linewidth, height=0.05\linewidth]{Table_figures/bar39.pdf} & -1.24 & -1.2 \\ 
  Microbiotheria & family & 1/1 & \includegraphics[width=0.20\linewidth, height=0.05\linewidth]{Table_figures/bar43.pdf} &   &   \\ 
  Microbiotheria & genus & 1/1 & \includegraphics[width=0.20\linewidth, height=0.05\linewidth]{Table_figures/bar44.pdf} &   &   \\ 
  Microbiotheria & species & 1/1 & \includegraphics[width=0.20\linewidth, height=0.05\linewidth]{Table_figures/bar45.pdf} &   &   \\ 
  Monotremata & family & 2/2 & \includegraphics[width=0.20\linewidth, height=0.05\linewidth]{Table_figures/bar46.pdf} &   &   \\ 
  Monotremata & genus & 2/3 & \includegraphics[width=0.20\linewidth, height=0.05\linewidth]{Table_figures/bar47.pdf} & -0.68 & -0.69 \\ 
  Monotremata & species & 2/4 & \includegraphics[width=0.20\linewidth, height=0.05\linewidth]{Table_figures/bar48.pdf} & -1.01 & -1 \\ 
  Notoryctemorphia & family & 1/1 & \includegraphics[width=0.20\linewidth, height=0.05\linewidth]{Table_figures/bar49.pdf} &   &   \\ 
  Notoryctemorphia & genus & 1/1 & \includegraphics[width=0.20\linewidth, height=0.05\linewidth]{Table_figures/bar50.pdf} &   &   \\ 
  Notoryctemorphia & species & 0/2 & \includegraphics[width=0.20\linewidth, height=0.05\linewidth]{Table_figures/bar51.pdf} &   &   \\ 
  Paucituberculata & family & 1/1 & \includegraphics[width=0.20\linewidth, height=0.05\linewidth]{Table_figures/bar52.pdf} &   &   \\ 
  Paucituberculata & genus & 3/3 & \includegraphics[width=0.20\linewidth, height=0.05\linewidth]{Table_figures/bar53.pdf} &   &   \\ 
  Paucituberculata & species & 5/5 & \includegraphics[width=0.20\linewidth, height=0.05\linewidth]{Table_figures/bar54.pdf} &   &   \\ 
  Peramelemorphia & family & 2/2 & \includegraphics[width=0.20\linewidth, height=0.05\linewidth]{Table_figures/bar55.pdf} &   &   \\ 
  Peramelemorphia & genus & 7/7 & \includegraphics[width=0.20\linewidth, height=0.05\linewidth]{Table_figures/bar56.pdf} &   &   \\ 
  Peramelemorphia & species & 16/18 & \includegraphics[width=0.20\linewidth, height=0.05\linewidth]{Table_figures/bar57.pdf} & -0.14 & 0.91 \\ 
  Perissodactyla & family & 3/3 & \includegraphics[width=0.20\linewidth, height=0.05\linewidth]{Table_figures/bar58.pdf} &   &   \\ 
  Perissodactyla & genus & 6/6 & \includegraphics[width=0.20\linewidth, height=0.05\linewidth]{Table_figures/bar59.pdf} &   &   \\ 
  Perissodactyla & species & 10/16 & \includegraphics[width=0.20\linewidth, height=0.05\linewidth]{Table_figures/bar60.pdf} & -0.1 & -2.77 \\ 
  Pholidota & family & 1/1 & \includegraphics[width=0.20\linewidth, height=0.05\linewidth]{Table_figures/bar61.pdf} &   &   \\ 
  Pholidota & genus & 1/1 & \includegraphics[width=0.20\linewidth, height=0.05\linewidth]{Table_figures/bar62.pdf} &   &   \\ 
  Pholidota & species & 4/8 & \includegraphics[width=0.20\linewidth, height=0.05\linewidth]{Table_figures/bar63.pdf} & 1.14 & 0.97 \\ 
  Pilosa & family & 4/5 & \includegraphics[width=0.20\linewidth, height=0.05\linewidth]{Table_figures/bar64.pdf} & 2.01 & 1.96 \\ 
  Pilosa & genus & 4/5 & \includegraphics[width=0.20\linewidth, height=0.05\linewidth]{Table_figures/bar65.pdf} & -0.91 & 0.36 \\ 
  \textbf{Pilosa} & \textbf{species} & \textbf{5/29} & \includegraphics[width=0.20\linewidth, height=0.05\linewidth]{Table_figures/bar66.pdf} & \textbf{1.18} & \textbf{2.35**} \\ 
  Primates & family & 15/15 & \includegraphics[width=0.20\linewidth, height=0.05\linewidth]{Table_figures/bar67.pdf} &   &   \\ 
  Primates & genus & 48/68 & \includegraphics[width=0.20\linewidth, height=0.05\linewidth]{Table_figures/bar68.pdf} & -0.37 & -1.39 \\ 
  Primates & species & 64/351 & \includegraphics[width=0.20\linewidth, height=0.05\linewidth]{Table_figures/bar69.pdf} & -0.66 & -1.4 \\ 
  Proboscidea & family & 1/1 & \includegraphics[width=0.20\linewidth, height=0.05\linewidth]{Table_figures/bar70.pdf} &   &   \\ 
  Proboscidea & genus & 2/2 & \includegraphics[width=0.20\linewidth, height=0.05\linewidth]{Table_figures/bar71.pdf} &   &   \\ 
  Proboscidea & species & 2/3 & \includegraphics[width=0.20\linewidth, height=0.05\linewidth]{Table_figures/bar72.pdf} & -0.67 & -0.72 \\ 
  Rodentia & family & 18/32 & \includegraphics[width=0.20\linewidth, height=0.05\linewidth]{Table_figures/bar73.pdf} & 0.66 & -0.95 \\ 
  \textbf{Rodentia} & \textbf{genus} & \textbf{82/450} & \includegraphics[width=0.20\linewidth, height=0.05\linewidth]{Table_figures/bar74.pdf} & \textbf{-1.81} & \textbf{1.7*} \\ 
  \textbf{Rodentia} & \textbf{species} & \textbf{90/2094} & \includegraphics[width=0.20\linewidth, height=0.05\linewidth]{Table_figures/bar75.pdf} & \textbf{2.66**} & \textbf{2.36**} \\ 
  Scandentia & family & 2/2 & \includegraphics[width=0.20\linewidth, height=0.05\linewidth]{Table_figures/bar76.pdf} &   &   \\ 
  Scandentia & genus & 2/5 & \includegraphics[width=0.20\linewidth, height=0.05\linewidth]{Table_figures/bar77.pdf} & -0.77 & -0.76 \\ 
  Scandentia & species & 3/20 & \includegraphics[width=0.20\linewidth, height=0.05\linewidth]{Table_figures/bar78.pdf} & -2 & -0.8 \\ 
  Sirenia & family & 2/2 & \includegraphics[width=0.20\linewidth, height=0.05\linewidth]{Table_figures/bar79.pdf} &   &   \\ 
  Sirenia & genus & 2/2 & \includegraphics[width=0.20\linewidth, height=0.05\linewidth]{Table_figures/bar80.pdf} &   &   \\ 
  Sirenia & species & 4/4 & \includegraphics[width=0.20\linewidth, height=0.05\linewidth]{Table_figures/bar81.pdf} &   &   \\ 
  Soricomorpha & family & 3/4 & \includegraphics[width=0.20\linewidth, height=0.05\linewidth]{Table_figures/bar82.pdf} & -0.98 & -0.97 \\ 
  \textbf{Soricomorpha} & \textbf{genus} & \textbf{19/43} & \includegraphics[width=0.20\linewidth, height=0.05\linewidth]{Table_figures/bar83.pdf} & \textbf{7.07**} & \textbf{2.64**} \\ 
  \textbf{Soricomorpha} & \textbf{species} & \textbf{21/392} & \includegraphics[width=0.20\linewidth, height=0.05\linewidth]{Table_figures/bar84.pdf} & \textbf{10.17**} & \textbf{3.36**} \\ 
  Tubulidentata & family & 1/1 & \includegraphics[width=0.20\linewidth, height=0.05\linewidth]{Table_figures/bar85.pdf} &   &   \\ 
  Tubulidentata & genus & 1/1 & \includegraphics[width=0.20\linewidth, height=0.05\linewidth]{Table_figures/bar86.pdf} &   &   \\ 
  Tubulidentata & species & 1/1 & \includegraphics[width=0.20\linewidth, height=0.05\linewidth]{Table_figures/bar87.pdf} &   &   \\ 
   \hline
\hline
\label{Table_results}
\end{longtable}


Among the mammalian orders containing OTUs with no available cladistic data, only @two orders, Carnivora and Chiroptera, have OTUs with cladistic data that are significantly phylogenetically clustered at the species- and the genus-level (Table~\ref{Table_data_structure}).

Two contrasting results are shown in Figure \ref{Figure_example_coverage} with randomly distributed OTUs with cladistic data in Primates (Figure \ref{Figure_example_coverage}A) and phylogenetically clustered OTUs with cladistic data in Carnivora (mainly Canidae; Figure \ref{Figure_example_coverage}B).

\begin{figure}[!htbp]
\centering
    \includegraphics[width=1\textwidth]{example_coverage.pdf}
\caption{Phylogenetic distribution of species with available cladistic data across two mammalian orders (A: Primates; B: Carnivora).
Edges are colored in grey when no cladistic data is available for a species or in blue when data is available.}
\label{Figure_example_coverage}
\end{figure}

%---------------------------------------------
%
%       DISCUSSION
%
%---------------------------------------------

\section{Discussion}
Our results show that although phylogenetic relationships among living mammals are well-resolved \citep[e.g.][]{FritzTree,meredithimpacts2011,May-Collado-PeerJ}, most of the data used to build these phylogenies is molecular, and very little cladistic data is available for living mammals compared to fossil mammals \citep[e.g.][]{O'Leary08022013,ni2013oldest}.
This has implications for building Total Evidence phylogenies containing both living and fossil mammals, as without sufficient cladistic data for living species, fossil placements in these trees are very uncertain \citep{GuillermeCooper}.
However, cladistic data coverage in living mammals varies across taxonomic levels and in its phylogenetic distribution.
Higher taxonomic levels are always better sampled than lower ones and within these taxonomic levels, the available data is mostly randomly distributed across the phylogeny, apart from in Carnivora and Chiroptera.
% NC: Add a sentence saying this means things might not be quite so bad? Need to justify the "However". however what?

The number of living mammalian taxa with no available cladistic data was surprisingly high at the species-level: only @4 out of @28 orders have a high coverage of taxa with available cladistic data (and two of the 28 orders are monospecific!).
This high coverage threshold of 75\% of taxa with available cladistic data represents the minimum amount of data required before missing data has a significant effect on the topology of Total Evidence trees \citep{GuillermeCooper}.
Beyond this threshold, there is considerable displacement of wildcard taxa (\textit{sensu} \citep{kearneyfragmentary2002}) and decreases in clade conservation \citep{GuillermeCooper}.
Therefore we expect a high probability of topological artefacts for the placement of fossil taxa at the species-level in most mammalian orders.
However, data coverage seems to be less of an issue at higher taxonomic levels (i.e. genus- and family-level).
This point is important from a practical point of view because of the slight discrepancy between the neontological and palaeontological concept of species.
While neontological species are described using morphology, genetic distance, spatial distribution and even behaviour \citep[e.g.][]{kellymolecular2014}, palaeontological data can be based only on morphological, spatial and temporal data \citep[e.g.][]{ni2013oldest}.
Because of this difference between neontological species (i.e. reproductive isolates) and paleontological species (i.e. morpho-species), most palaeontological studies are using the genus-level as their smallest OTU \citep[e.g.][]{ni2013oldest,O'Leary08022013}.
Thus data availability at the genus-level in living mammals should be our primary concern when aiming to build phylogenies of living and fossil taxa.

When only a few species with cladistic data are available, the ideal scenario is for them to be phylogenetically overdispersed (i.e. that there is data for at least every sub-clade) to maximize the possibilities of a fossil branching from the right clade.
The second best scenario is that species with cladistic data are randomly distributed across the phylogeny. 
In this scenario we expected no special bias in the placement of the fossil \citep{GuillermeCooper}, it is therefore encouraging that for all orders except Carnivora and Chiroptera, species with cladistic data were randomly distributed across the phylogeny of each order.
The worst case scenario for fossil placement is that species with cladistic data are phylogenetically clustered. 
In this situation we expect two major biases to occur: first, the fossil will not be able to branch within a clade containing no data, and second, the fossil will have a higher probability, at random, of branching within the clade containing most of the available data.
This means that fossils with uncertain phylogenetic affinities (\textit{incertae sedis}) will have a higher probability of branching within the most sampled clade just by chance.
Our results suggest that this may be an issue in Carnivora and Chiroptera. 
For example, a Carnivora fossil will be unable to branch in the Herpestidae that has no species with cladistic data, and will also have more chance to branch, randomly, within the Canidae clade than any other clade in Carnivora (Figure \ref{Figure_example_coverage}-B).
Thus placements of Carnivoran fossils should be viewed with suspicion. % NC: OK gotta be a better way to say that.

%Caveats
In this study, we treated all cladistic matrices as equal in a similar way to molecular matrices. 
For example, if matrix A contained 100 characters for taxa X and Y, and matrix B contained 50 different characters for taxa X and Z, we assumed that both matrices can be combined in a supermatrix containing 150 independent characters for taxon X, 100 for taxon Y and 50 for taxon Z.
Unfortunately, cladistic data cannot always be treated in this way because some characters may overlap.
For example, if matrix A has a character coding for the shape of a particular morphological feature and matrix B has a character coding for the presence of this morphological feature and a second character coding for its shape, then these three characters are non-independent compound characters \citep{Brazeau2011}.
However, in reasonably sized matrices (\textgreater 100 characters; \citealp{GuillermeCooper,harrisonamong-character2014}) it is more likely that a number of characters are consistently conserved among the different matrices (e.g. within the Primate literature, the character \textit{p7} - the size of the $4^{th}$ lower premolar paraconid - has been used consistently for \textgreater 15 years; \citealp[e.g.][]{ross1998phylogenetic,seiffert2003fossil,kay2008anatomy,boyer2010astragalar,marivaux2013djebelemur}). % NC: Remove some of these!!!
A conservative approach to avoid compound characters would be to select only the most recent matrix for each group, but this would result in the loss of a lot of data.

%Let's code some data
Following the recommendations in \citep{GuillermeCooper}, to improve the accuracy of the topology of Total Evidence trees containing both living and fossil taxa, one should code cladistic characters for as many living species possible. 
Because the data for living mammals is usually readily available in natural history collections, we propose that an increased effort be put into coding morphological characters from living species, possibly by engaging in collaborative data collection projects through web portals such as \textit{Morphobank} \citep{morphobank}.

% NC: OK needs a summing up sentence!!!

%Biology letters various stuff
\section{Ethics statement}
\section{Data accessibility statement}
All data and analysis code is available on GitHub.
\section{Authors' Contributions}
T.G. and N.C conceived and designed the experiments. T.G. performed the experiments and analysed the data. T.G. and N.C. contributed to the writing of the manuscript. All authors approved the final version of the manuscript.
\section{Competing Interests}
We have no competing interests.
\section{Acknowledgements}
David Bapst, Graeme Lloyd, Nick Matzke and April Wright.
\section{Funding statement}
This work was funded by a European Commission CORDIS Seventh Framework Programme (FP7) Marie Curie CIG grant (proposal number: 321696).

\bibliographystyle{sysbio}
\bibliography{References}

%\section{SOM}
\newcommand{\beginsupplement}{%
    \setcounter{table}{0}
    \renewcommand{\thetable}{S\arabic{table}}%
    \setcounter{figure}{0}
    \renewcommand{\thefigure}{S\arabic{figure}}%
}
\beginsupplement
%
\noindent{\Large \bf Supplementary Material}

\bigskip

\section{Data collection}
1- Data collection: key words, clade (ordinal) metacharacters, Google Search terms, Google Search protocol, Google Search rarefaction curve.

\subsection{search terms}
The searched mammalian order terms are available in $search_terms_latin.txt$ or $search_terms_meta.txt$. The file containing the meta names is the Latin name files but with replacing the latin suffixes ([ia|ata|ea|a]) by a joker character (*) and by replacing the first letter by a upper/lower case meta character (e.g. [Aa]).
Mammalia; Monotremata; Marsupialia; Placentalia; Macroscelidea; Afrosoricida; Tubulidentata; Hyracoidea; Proboscidea; Sirenia; Pilosa; Cingulata; Scandentia; Dermoptera; Primates; Lagomorpha; Rodentia; Erinaceomorpha; Soricomorpha; Cetacea; Artiodactyla; Cetartiodactyla; Chiroptera; Perissodactyla; Pholidota; Carnivora; Didelphimorphia; Paucituberculata; Microbiotheria; Dasyuromorphia; Peramelemorphia; Notoryctemorphia and Diprotodontia.

\subsubsection{Ross Mounce data set}
I selected all the matrices containing at least one of the mammalian orders names from Ross Mounce GitHub $cladistic-data/nexus_files$ repository (accessed on the 02/12/2014).

\subsubsection{Graeme Lloyd}
Selecting the downloadable matrices

\subsubsection{Morphobank}
\textit{order}

\subsubsection{Google scholars}
20 first results since 2010 with the following exact key words:

\texttt{\textit{order} ("morphology" OR "morphological" OR "cladistic") AND characters matrix paleontology phylogeny}

We selected only the 20 first results per search term because only the 50 first of the 660 papers added 425 living OTUs.

\begin{figure}[!htbp]
\centering
    \includegraphics[width=1\textwidth]{Supplementary/Supp_figure_google_searches.pdf}
\caption{Google searches additional OTUs rarefaction curve. The x axis represent the number of google scholar matches (papers, books or abstracts) and the y axis represents the cumulative number of additional living OTUs per google scholar match.}
\label{Supp_figure_google_searches}
\end{figure}


\subsection{Wrong Bionmial names and typos}
I fixed the wrong bionomial names format (e.g. H. sapiens) into the correct ones (e.g. Homo sapiens) manually using the abreviation list in the concerned publications. We then applied our taxonomic matching algorithm to classify the OTUs as either living or fossil.

\begin{figure}[!htbp]
\centering
    \includegraphics[width=1\textwidth]{Supplementary/Supp_figure_Taxonomic_algorithm.pdf}
\caption{Taxonomic matching algorithm used in this study. For each matrix, each operational taxonomic units (OTU) is matched with the super tree from Fritz 2009. If the OTU matches, then it is classified as living. Else it is matched with the Wilson \& Reeders 2005 taxonomy list. If the OTU matches, then it is classified as living. Else it is matched with the Paleo Database list of mammals. If the OTU matches, then it is classified as fossil. Else it is ignored.}
\label{Supp_figure_Taxonomic_algorithm}
\end{figure}

\bibliographystyle{vancouver}
\bibliography{Supp_References}


\section{Data structure}
%Tables
%Figures

\section{Supplementary results}



%END
\end{document}
