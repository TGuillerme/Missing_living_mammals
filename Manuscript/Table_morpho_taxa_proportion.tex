% latex table generated in R 3.1.1 by xtable 1.7-3 package
% Fri Feb 27 10:08:44 2015

% NC: Might be good to have genus and species with small letters
% NC: I switched OTUs for taxa so you don't have to define it. If this is incorrect we need to change it back and define in the legend.
% NC: Also taxa are arranged according to higher level taxonomy I see. Might be good to separate that to make the table easier to read: e.g.
%Marsupalia
%Didelphimorphia & Family & 1/1 & 100 \\ 
% Otherwise it might be better in alphabetical order? Easier to find the clade you want that way. Same for the other table too.

\begin{longtable}{llll}
\caption{Number of taxa with available cladistic data for mammalian orders at three taxonomic levels. Orders in bold have \textgreater 75\% of missing data at each taxonomic level, i.e. ``poor'' data coverage. 
% NC: What about good data coverage? Should we not highlight them somehow too?
% Note that it is possible that more data is available at a higher taxonomic level (Genus $>$ Species) since if the species name for an OTU was not or miss specified, we still counted the OTU for higher taxonomic level analysis. % NC: Isn't that obvious? Maybe put in text???
} \\ 
  \hline
Order & Taxonomic level & Proportion of taxa with data & \% taxa with data \\ 
  \hline
Monotremata & Family & 2/2 & 100 \\ 
  Monotremata & Genus & 2/3 & 66.67 \\ 
  Monotremata & Species & 2/4 & 50 \\ 
  Didelphimorphia & Family & 1/1 & 100 \\ 
  Didelphimorphia & Genus & 16/16 & 100 \\ 
  Didelphimorphia & Species & 40/84 & 47.62 \\ 
  Paucituberculata & Family & 1/1 & 100 \\ 
  Paucituberculata & Genus & 2/3 & 66.67 \\ 
  Paucituberculata & Species & 2/5 & 40 \\ 
  Microbiotheria & Family & 1/1 & 100 \\ 
  Microbiotheria & Genus & 1/1 & 100 \\ 
  Microbiotheria & Species & 1/1 & 100 \\ 
  Notoryctemorphia & Family & 1/1 & 100 \\ 
  Notoryctemorphia & Genus & 1/1 & 100 \\ 
  \textbf{Notoryctemorphia} & \textbf{Species} & \textbf{0/2} & \textbf{0} \\ 
  Dasyuromorphia & Family & 2/2 & 100 \\ 
  Dasyuromorphia & Genus & 7/22 & 31.82 \\ 
  \textbf{Dasyuromorphia} & \textbf{Species} & \textbf{8/64} & \textbf{12.5} \\ 
  Peramelemorphia & Family & 2/2 & 100 \\ 
  Peramelemorphia & Genus & 7/7 & 100 \\ 
  Peramelemorphia & Species & 16/18 & 88.89 \\ 
  Diprotodontia & Family & 9/11 & 81.82 \\ 
  Diprotodontia & Genus & 20/38 & 52.63 \\ 
  \textbf{Diprotodontia} & \textbf{Species} & \textbf{16/126} & \textbf{12.7} \\ 
  Afrosoricida & Family & 2/2 & 100 \\ 
  Afrosoricida & Genus & 17/17 & 100 \\ 
  Afrosoricida & Species & 23/42 & 54.76 \\ 
  Macroscelidea & Family & 1/1 & 100 \\ 
  Macroscelidea & Genus & 4/4 & 100 \\ 
  Macroscelidea & Species & 5/15 & 33.33 \\ 
  Tubulidentata & Family & 1/1 & 100 \\ 
  Tubulidentata & Genus & 1/1 & 100 \\ 
  Tubulidentata & Species & 1/1 & 100 \\ 
  Hyracoidea & Family & 1/1 & 100 \\ 
  Hyracoidea & Genus & 1/3 & 33.33 \\ 
  Hyracoidea & Species & 1/4 & 25 \\ 
  Proboscidea & Family & 1/1 & 100 \\ 
  Proboscidea & Genus & 1/2 & 50 \\ 
  Proboscidea & Species & 1/3 & 33.33 \\ 
  Sirenia & Family & 2/2 & 100 \\ 
  Sirenia & Genus & 2/2 & 100 \\ 
  Sirenia & Species & 2/4 & 50 \\ 
  Cingulata & Family & 1/1 & 100 \\ 
  Cingulata & Genus & 8/9 & 88.89 \\ 
  \textbf{Cingulata} & \textbf{Species} & \textbf{6/25} & \textbf{24} \\ 
  Pilosa & Family & 3/5 & 60 \\ 
  Pilosa & Genus & 3/5 & 60 \\ 
  \textbf{Pilosa} & \textbf{Species} & \textbf{3/29} & \textbf{10.35} \\ 
  Scandentia & Family & 2/2 & 100 \\ 
  Scandentia & Genus & 2/5 & 40 \\ 
  \textbf{Scandentia} & \textbf{Species} & \textbf{2/20} & \textbf{10} \\ 
  Dermoptera & Family & 1/1 & 100 \\ 
  Dermoptera & Genus & 1/2 & 50 \\ 
  Dermoptera & Species & 1/2 & 50 \\ 
  Primates & Family & 15/15 & 100 \\ 
  Primates & Genus & 48/68 & 70.59 \\ 
  \textbf{Primates} & \textbf{Species} & \textbf{56/351} & \textbf{15.95} \\ 
  Rodentia & Family & 10/32 & 31.25 \\ 
  \textbf{Rodentia} & \textbf{Genus} & \textbf{20/451} & \textbf{4.43} \\ 
  \textbf{Rodentia} & \textbf{Species} & \textbf{10/2095} & \textbf{0.48} \\ 
  Lagomorpha & Family & 1/2 & 50 \\ 
  \textbf{Lagomorpha} & \textbf{Genus} & \textbf{1/12} & \textbf{8.33} \\ 
  \textbf{Lagomorpha} & \textbf{Species} & \textbf{1/86} & \textbf{1.16} \\ 
  Erinaceomorpha & Family & 1/1 & 100 \\ 
  Erinaceomorpha & Genus & 10/10 & 100 \\ 
  Erinaceomorpha & Species & 21/22 & 95.45 \\ 
  Soricomorpha & Family & 3/4 & 75 \\ 
  Soricomorpha & Genus & 19/43 & 44.19 \\ 
  \textbf{Soricomorpha} & \textbf{Species} & \textbf{19/392} & \textbf{4.85} \\ 
  Chiroptera & Family & 13/18 & 72.22 \\ 
  Chiroptera & Genus & 68/202 & 33.66 \\ 
  \textbf{Chiroptera} & \textbf{Species} & \textbf{108/1054} & \textbf{10.25} \\ 
  Pholidota & Family & 1/1 & 100 \\ 
  Pholidota & Genus & 1/1 & 100 \\ 
  Pholidota & Species & 3/8 & 37.5 \\ 
  Carnivora & Family & 11/15 & 73.33 \\ 
  \textbf{Carnivora} & \textbf{Genus} & \textbf{30/125} & \textbf{24} \\ 
  \textbf{Carnivora} & \textbf{Species} & \textbf{42/283} & \textbf{14.84} \\ 
  Perissodactyla & Family & 3/3 & 100 \\ 
  Perissodactyla & Genus & 6/6 & 100 \\ 
  Perissodactyla & Species & 7/16 & 43.75 \\ 
  Cetartiodactyla & Family & 20/21 & 95.24 \\ 
  Cetartiodactyla & Genus & 76/128 & 59.38 \\ 
  Cetartiodactyla & Species & 106/311 & 34.08 \\ 
   \hline
\hline
\label{Table_morpho_taxa_proportion}
\end{longtable}
