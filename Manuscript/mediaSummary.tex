\documentclass{article}


\begin{document}

Scientists use family trees to understand changes in biodiversity through time.
But the trees often only use living species, ignoring extinct groups like mammoths.Therefore we must place living and extinct species in the same trees.
We can do so using the DNA from living species and anatomical data from living and extinct species
However, we found that few living mammals have enough available anatomical data to determine how they are related to extinct species
We suggest that we must collect more anatomical data from living species in museum collections to efficiently create family trees of living and extinct species.

\end{document} 
